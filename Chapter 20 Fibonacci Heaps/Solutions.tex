\documentclass[a4paper, fleqn]{article}
\usepackage{amsmath}
\usepackage{clrscode}
\usepackage{hyperref}
\setlength\mathindent{0em}
\parskip = 7bp

\title{Solutions for Chapter 20}
\author{Zhixiang Zhu
\\\href{mailto:zzxiang21cn@hotmail.com}{zzxiang21cn@hotmail.com}}
\date{January 25, 2011}
\begin{document}

\maketitle

\section*{Solution to Exercise 20.2-3}

A Fibonacci heap remains a collection of unordered binomial trees
after any mergeable-heap operation is performed. Therefore, $D(n)$ is
at most $\lfloor \lg n \rfloor$.





\section*{Solution to Exercise 20.2-4}

The inserion and union operations without consolidation both take
$O(1)$ time. The consolidation operation takes $O(D(n[H]) + t(H))$
time, $H$ is the input heap of consolidation. Therefore, the new
insertion and union operations both take $O(D(n[H]) + t(H))$ actual
time, where $H$ is the heap resulted from the insertion or union
operation. Since the amortized costs of the original insertion and
union operations are already $O(1)$, the new operations do no good to
it. Actually they make the actual running time even worse.






\section*{Solution to Exercise 20.3-1}

If $x$ is a marked child of some root, then $x$ will become a marked
root when the parent of $x$ is deleted and $x$ does not become a child
of any other node.






\section*{Solution to Exercise 20.3-2}

Suppose that we consecutively invoke \proc{Fib-Heap-Decrease-Key} $n$
times on a Fibonacci heap of $n$ elements. As we have known, the only
non-constant time operation in \proc{Fib-Heap-Decrease-Key} is
\proc{Cascading-Cut}. Each recursive \proc{Cascading-Cut} (those which
are neither the top nor the bottom in the invocation stack) invokes
\proc{Cut} on a node, with the node being the $x$ input of \proc{Cut}.
Therefore, the only non-constant time operation in
\proc{Fib-Heap-Decrease-Key} is actually the multiple \proc{Cut}
procedures. Since the $x$ input node of \proc{Cut} will become a root
and \proc{Cut} won't be invoked on a root with the root being $x$,
\proc{Cut} is invoked at most $n$ times in total. Therefore, the total
time of the $n$ invocations of \proc{Fib-Heap-Decrease-Key} is $O(n)$,
and the amortized time of \proc{Fib-Heap-Decrease-Key} is $O(n) / n =
O(1)$.






\section*{Solution to Exercise 20.4-1}

After we create an empty Fibonacci heap, we insert a node into the
heap. After the insertion, we perform a sequence of the following
operations:

\noindent
I2, E, I2, E, I2, E, ...

\noindent
where I2 stands for inserting two nodes whose keys are smaller than
the currnet minimum key in the heap, that is, suppose the current
minimum key in the heap is $A$, then we insert two nodes whose keys
are $B$ and $C$, where $B < A$ and $C < A$; E stands for extracting
the node with the minimum key.

After $n-1$ I2 and $n-1$ E operations, we will get a Fibonacci heap in
which there's only one tree, which is just a linear chain of $n$
nodes.






\section*{Solution to Execise 20.4-2}

As long as $k$ is a positive integer, we will have $D(n) = O(\lg
n)$. The proof is given as follows.

If a node $x$ is cut as soon as it loses its $k'$th child, then the
last sentence of lemma $20.1$ in the text will be changed to: ``Then,
$\id{degree}[y_i] \geq 0$ for $i = 1, 2, \ldots, k' - 1$ and
$\id{degree}[y_i] \geq i - k'$ for $i = k', k' + 1, \ldots, k$.''
Therefore, the the first half of the proof of lemma 20.3 will also be
changed to
\begin{align*}
\mbox{size}(x) &\geq s_k \\
               &\geq k' + \sum_{i = k'}^k s_{\id{degree}[y_i]} \\
	       &\geq k' + \sum_{i = k'}^k s_{i - k'}.
\end{align*}
(to be continued...)






\section*{Solution to Problem 20-1}

\begin{enumerate}
\renewcommand{\labelenumi}{\itshape \bfseries \alph{enumi}.}

\item  % a

The $p$ fields of $x$'s children should be set to \const{nil} in line
7, which takes $\Theta(\id{degree}[x])$ actual time.

\item  % b

$O(\id{degree}[x] + c)$.

\item  % c

The potential of $H'$ is at most $t(H) + c + \id{degree}[x] + 2(m(H) -
c + 2)$.

\item  % d

The change in potential is at most
\begin{align*}
&(t(H) + c + \id{degree}[x] + 2(m(H) - c + 2)) - (t(H) + 2m(H)) \\
&\qquad = \id{degree}[x] + 4 - c.
\end{align*}

\end{enumerate}






\section*{Solution to Problem 20-2}

\begin{enumerate}
\renewcommand{\labelenumi}{\itshape \bfseries \alph{enumi}.}

\item  % a

Here's the \proc{Fib-Heap-Change-Key} procedure:

\begin{codebox}
\Procname{$\proc{Fib-Heap-Change-Key}(H, x, k)$}
\li \If $k > \id{key}[x]$
\li   \Then
        $\id{key}[x] \gets k$
\li     \For each child $y$ in the child list of $x$
	\label{li:20-2a-for-1-begin}
\li       \Do
	    \If $\id{key}[y] < \id{key}[x]$
\li           \Then
	        $\proc{Cut}(H, y, x)$
\li             $\proc{Cascading-Cut}(H, x)$
	      \End
	  \End
	\label{li:20-2a-for-1-end}
\li     \If $x = \id{min}[H]$
\li       \Then
            $\proc{Consolidate}(H)$
\li         \For each node $z$ in the root list of $H$
            \label{li:20-2a-for-2-begin}
\li           \Do
                \If $\id{key}[z] < \id{key}[\id{min}[H]]$
\li               \Then
                    $\id{min}[H] \gets z$
                  \End
              \End
            \label{li:20-2a-for-2-end}
          \End
\li   \Else
        $\proc{Fib-Heap-Decrease-Key}(H, x, k)$
      \End
\end{codebox}

Let $H$ denote the Fibonacci heap just prior to the
\proc{Fib-Heap-Change-Key} operation.

Now let's determine the amortised cost of \proc{Fib-Heap-Change-Key}
in the case of $k > \id{key}[x]$.

If $x = \id{min}[H]$, there will be no recursive calls of
\proc{Cascading-Cut} since $p[x] = \const{nil}$. Therefore, the \For
loop of lines \ref{li:20-2a-for-1-begin}--\ref{li:20-2a-for-1-end}
contributes an actual cost of $O(D(n))$, since there're at most $D(n)$
children of $x$. According to the analysis in section 20.2 of the
textbook, the \proc{Consolidate} procedure contributes an actual cost
$O(D(n) + t(H))$, so as the \For loop of lines
\ref{li:20-2a-for-2-begin}--\ref{li:20-2a-for-2-end}. Therefore, the
actual cost of \proc{Fib-Heap-Change-Key} when $x = \id{min}[H]$ is
$O(D(n) + t(H))$.

The potential before changing the key is $t(H) + 2m(H)$, and the
potential afterward is at most $(D(n) + 1) + 2m(H)$, since at most
$D(n) + 1$ roots remain and no nodes become marked during the
operation. The amortised cost when $x = \id{min}[H]$ is thus at most
\begin{align*}
&O(D(n) + t(H)) + ((D(n) + 1) + 2m(H)) - (t(H) + 2m(H)) \\
&\qquad = O(D(n)) + O(t(H)) - t(H) \\
&\qquad = O(D(n)),
\end{align*}
since we can scale up the units of potential to dominate the constant
hidden in $O(t(H))$.

If $x \neq \id{min}[H]$, the only cost occurs in the \For loop of
lines \ref{li:20-2a-for-1-begin}--\ref{li:20-2a-for-1-end}, in which
case the actual cost is $O(D(n) + c)$, where $c$ is the number of
times \proc{Cascading-Cut} is recursively called.

We next compute the change in potential. Each recursive call of \\
\proc{Cascading-Cut}, except for the last one, cuts a marked node and
clears the mark bit. The children of $x$ may also be cut. Afterward,
there are at most $t(H) + D(n) + c - 1$ trees (the original $t(H)$
trees, $D(n)$ trees produced by cutting $x$'s children, and $c - 1$
trees produced by cascading cuts), and at most $m(H) - c + 2$ marked
nodes ($c - 1$ were unmarked by cascading cuts and the last call of
\proc{Cascading-Cut} may have marked a node). The change in potential
is therefore at most
\begin{align*}
&((t(H) + D(n) + c - 1) + 2(m(H) - c + 2)) - (t(H) + 2m(H)) \\
&\qquad= D(n) + 4 - c
\end{align*}
Thus, the amortised cost of \proc{Fib-Heap-Change-Key} when $x \neq
\id{min}[H]$ is at most
\[O(D(n) + c) + D(n) + 4 - c = O(D(n)),\]
since we can scale up the units of potential to dominate the constant
hidden in $O(c)$.

Therefore, the amortised cost of \proc{Fib-Heap-Change-Key} for the
case in which $k$ is greater than $\id{key}[x]$ is $O(D(n))$. Because
a node is cut as soon as it loses two children, lemma $20.1$ in the
textbook still holds, so do lemma 20.3 and corollary 20.4. Thus, the
amortised cost of \proc{Fib-Heap-Change-Key} for the case in which $k$
is greater than $\id{key}[x]$ is $O(\lg n)$.

According to section 20.3 of the textbook, the amortised cost is
$O(1)$ for the case in which $k$ is equal to or less than
$\id{key}[x]$.



\item  % b

We can modify the data structure to maintain a list of leaves. As long
as the number of deleted nodes is smaller than $\min(r, n[H])$, we
will delete a leaf from the list. Each time a leaf is deleted, the
\proc{Cascading-Cut} procedure must be called on its parent to keep
lemma $20.1$ in the textbook true, so that the amortised running time
of any other Fibonacci-heap operations won't be changed. Also, each
time a leaf is added or deleted, the leaf list must be updated. The
potential function doesn't need to be modified. The amortised cost of
$\proc{Fib-Heap-Prune}(H, r)$ is $O(\min(r, n[H]))$.

\end{enumerate}

\end{document}
