\documentclass[a4paper, fleqn]{article}
\usepackage{clrscode}
\setlength\mathindent{0em}
\parskip = 7bp
\begin{document}

\section*{Solution to Exercise 19.1-3}

Assume that in any $B_{k-1}$, a node at depth $i$ has $j = k-1-i$ 1's
in its binary representation. When two $B_{k-1}$ form a $B_k$, each
node label in the right $B_{k-1}$ is increased by $2^{k-1}$, since the
left $B_{k-1}$ contains $2^{k-1}$ nodes. This will set the $(k-1)$-th
bit of the label of each node in the right $B_{k-1}$ to 1, so that one
more 1 appears in its binary representation. Also, the depth of each
node in the left $B_{k-1}$ is increased by 1. Thus, the nodes in depth
$i$ of the right $B_{k-1}$ will be at the same level in $B_k$ with the
nodes in depth $i-1$ of the left $B_{k-1}$, except that the root of
the right $B_{k-1}$ becomes the root of $B_k$, which has $k$ 1's in
its binary represantation. Therefore, in $B_k$, a node at depth $i$
has $j = k - i$ 1's in its binary representation.

The base case of the assumption in which $k - 1 = 0$ trivially holds
true. Therefore by induction, in any $B_k$, a node at depth $i$ has
$j = k - i$ 1's in its binary representation.

According to the third property of Lemma 19.1, there are
$\left(\begin{array}{c}k \\ k - j\end{array}\right)$ binary
$k$-strings that contain exactly $j$ 1's.

The last statement can also be shown by induction in the way described
above similarly.






\section*{Solution to Exercise 19.2-5}

If all the keys in the heap are $\infty$, then the procedure will
return \const{nil}. To make the procedure work in such a situation,
change the $<$ symbol in line 5 to $\leq$.






\section*{Solution to Exercise 19.2-10}

If $n$ is a power of 2, then the input heap contains only one binomial
tree, so that \proc{Binomial-Heap-Insert} and
\proc{Binomial-Heap-Minimum} runs in $O(1)$ time on any bionomial
heap of size $n$.






\section*{Solution to Problem 19-1}

\begin{enumerate}
\renewcommand{\labelenumi}{\itshape \bfseries \alph{enumi}.}
\stepcounter{enumi}  % a
\stepcounter{enumi}  % b
\stepcounter{enumi}  % c
\stepcounter{enumi}  % d
\stepcounter{enumi}  % e

\item  % f

See the \proc{Union} procedure below:

\begin{codebox}
\Procname{$\proc{Union}(H_1, H_2)$}
\li create a new node $r$
\li \If $\id{height}[H_1] = \id{height}[H_2]$
\li     \Then
            make $H_1$ the left child of $r$, make $H_2$ the
\zi         the right child of $r$, and make $r$ the root of
\zi         a new 2-3-4 heap $H$
\li         \Return $H$
        \End
\li assume without loss of generality that
    $\id{height}[H_1] > \id{height}[H_2]$
\li make $H_2$ the right child of $r$
\li $x \gets \mbox{the node which is in the rightmost path of } H_1$
\zi \>$\mbox{ and whose height is equal to } \id{height}[H_2] + 1$
\li \While $x$ is full
\li     \Do
            remove the rightmost subtree in $x$ from $x$ and
\zi         \>make it the left child of $r$, the rightmost element
\zi         \>in $x$ is also deleted
\li         create a new node $s$ and make $r$ be
\zi         \>the right child of $s$
\li         update $\id{small}[x]$, $\id{small}[r]$ and
              $\id{small}[s]$
\li         $r \gets s$
\li         \If $p[x] = \const{nil}$
\li             \Then
                    create a new node $y$ and
\zi                 \>make $x$ be the left child of $y$
\li                 $\id{small}[y] \gets \id{small}[x]$
\li                 $x \gets y$
                \End
\li         $x \gets p[x]$
        \End  % while
\li \If the left subtree of $x$ is not empty but
\zi     \>the right subtree is empty
\li     \Then
            delete the node $r$ and make its former right subtree
\zi         \>be the right subtree of $x$
\li     \Else
            merge $x$ and $r$ so that the only one element of $r$
\zi         \>be the rightmost element in $x$
        \End
\li \While $x \neq \const{nil}$
\li     \Do
            update $\id{small}[x]$
\li         $x \gets p[x]$
        \End
\li make $x$ the root of a new 2-3-4 heap $H$
\li \Return $H$
\end{codebox}

\end{enumerate}






\section*{Solution to Problem 19-2}

Each $V_i$ can be maintained by a binomial heap, within which each
node $v$ represents a vetex. $\id{key}[v]$ is the minimum weight of
any edge connecting $v$ to an ajacent vertex.

Each $E_i$ can also be maintained by a binomial heap, within which
each node $e$ represents an edge in $E_i$. $\id{key}[e]$ is the weight
of $e$. There are two satellite fields $v_1[e]$ and $v_2[e]$, which
point to the vetex node connected by the edge.

We also need to add an operation beyond the mergeable-heap operations
given in Figure 19.1 of the textbook for the vertex binomial heaps:
given a vertex node, we shall return the head of the heap it belongs
to. This operation is used to support the \If statement in line 9 of
\proc{MST}. In order to support this operation, we need to doubly link
the roots in a binomial heap. It takes $O(\lg n)$ time to find the
head for a binomial heap of $n$ elements.

Let's denote the number of vertices and the number of edges in the
graph by $V$ and $E$ respectively. The binomial heap constructions are
done in the \For loop of \proc{MST}. The vertex heaps are constructed
in $\Theta(E)$ time in total. Each $E_i$ can be constructed by one
\proc{Make-Heap} and multiple \proc{Insert} operations. So the edge
heaps are constructed in $O(E \lg E)$ time in total.

There are $2E$ edge nodes in total, and each node is extracted at most
once, which are done in line 7 of \proc{MST}, with each extraction
take $O(\lg E)$. Before the \If statement is performed, we need to
find the head of $V_i$ and $V_j$, which takes $O(\lg V)$ time. Merging
two vertex heaps every time in line 11 takes $O(\lg V)$, while merging
two edge heaps every time in line 12 takes $O(\lg E)$. Therefore, the
total running time is $O(E \lg V + E \lg E)$.

\end{document}
