\documentclass[a4paper, fleqn]{article}
\usepackage{clrscode}
\setlength\mathindent{0em}
\parskip=7bp
\begin{document}

\section{The worst case of \proc{Max-Heapify}}

\proc{Max-Heapify} contains a recursive call on a subtree rooted at one of the 
children of node $i$. The textbook says that ``the children's subtrees each 
have size at most $2n/3$---the worst case occurs when the last row of the tree 
is exactly half full''. Why $2n/3$?

Suppose that the worst case occurs, i.e. the last row of the tree is exactly 
half full and the height of tree is $h$. Then the subtree which is bigger has 
size $m = 2^{h-1+1} - 1 = 2^h - 1$, and the tree has size $n = 2^h - 1 + 
(2^{h+1} - 1 - 2^{h} + 1)/2 = 3 \cdot 2^{h-1} - 1$ from Exercise 6.1-1. 
Therefore, we have
\begin{eqnarray*}
\frac{m}{n} & = & \frac{2^h - 1}{3 \cdot 2^{h-1} - 1} \\
            & = & \frac{2 - \frac{1}{2^{h-1}}}{3 - \frac{1}{2^{h-1}}}.
\end{eqnarray*}
Thus, $m/n \approx 2/3$ for sufficiently large $n$.

\end{document}
