\documentclass[a4paper, fleqn]{article}
\usepackage{amsmath}
\usepackage{array}
\usepackage{clrscode}
\usepackage{enumitem}
\usepackage{hyperref}
\usepackage[margin=1in]{geometry}
\usepackage{color}
\setlength\mathindent{0em}
\parskip = 5bp

\title{Solutions for Chapter 34}
\author{Zhixiang Zhu  \\\href{mailto:zzxiang21cn@hotmail.com}{zzxiang21cn@hotmail.com}}

\begin{document}

\maketitle

\section*{Solution to Exercise 34.1-1}

The `if' part: If \proc{longest-path} $\in$ P, we can call \proc{longest-path} with $k =
|E|, |E| - 1, \ldots, 0$ until \proc{longest-path} returns 1 or $k = 0$ to get the length
of a longest simple path between $u, v$. There're at most $|E|$ such calls, so
\proc{longest-path-length} $\in$ P.

The `only if' part: If \proc{longest-path-length} $\in$ P, we can call
\proc{longest-path-length} in \proc{longest-path} to compare the length of a longest
simple path with $k$ to decide wether to return 0 or 1. So \proc{longest-path} $\in$ P.


\section*{Solution to Exercise 34.1-2}

The optimization problem is defined as the relation that associates each instance of an
undirected graph with a longest simple cycle in the graph. A related decision problem is
defined as \proc{longest-simple-cycle-length} = $\{ \langle G, k \rangle : G = (V, E)$ is
an undirected graph, $k \geq 0$ is an integer, and there exists a simple cycle in $G$
consisting of at least $k$ edges $\}$.

\end{document}
