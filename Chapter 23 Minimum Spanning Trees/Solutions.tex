\documentclass[a4paper, fleqn]{article}
\usepackage{amsmath}
\usepackage{array}
\usepackage{clrscode}
\usepackage{enumitem}
\usepackage{hyperref}
\usepackage[margin=1in]{geometry}
\usepackage{color}
\setlength\mathindent{0em}
\parskip = 5bp

\title{Solutions for Chapter 23}
\author{Zhixiang Zhu  \\\href{mailto:zzxiang21cn@hotmail.com}{zzxiang21cn@hotmail.com}}

\begin{document}

\maketitle

\section*{Solution to Exercise 23.1-2}

Let $V = \{ a, b, c \}$, $E = \{ (a, b), (b, c), (a, c) \}$, $w(a, b) = 1$, $w(b, c) = 2$, $w(a, c) = 3$, $A = \emptyset$ and $S = \{ b \}$. $(b, c)$ is a safe edge for $A$, but it is not a light edge for the cut.


\section*{Solution to Exercise 23.1-3}

If we remove $(u, v)$ from the minimum spanning tree it is in, the spanning tree will be disconnected and divided into two components. Let one of the two components be $S$, then $(u, v)$ must be a light edge crossing the cut $(S, V - S)$; otherwise we can add a light edge of the cut to the disconnected tree so that the newly formed spanning tree has a smaller total weight, contradicting that $(u, v)$ is in a minimum spanning tree.


\section*{Solution to Exercise 23.1-4}

Let's say we have a graph consisting of three vertices: $u, v, w$; and the weights of $(u, v), (v, w)$ and $(u, w)$ are all equal.


\section*{Solution to Exercise 23.1-5}

Because $e$ is on a cycle, any cut crossed by $e$ must also be crossed by some other edge on the cycle. If we build the minimum spanning tree by choosing light edges, $e$ will never be chosen, so that the constructed minimum spanning tree will not include $e$.


\section*{Solution to Exercise 23.1-7}

If all edge weights are positive, such a subset of edges must be acyclic; otherwise we can remove the edges who have bigger weights until the subset becomes acyclic to get a smaller total weight. Therefore, it must be a tree. If we allow some weights to be nonpositive, this would not hold. Example: suppose that a graph contains a cycle on which all the edge weights are negative, then such a subset must contains the cycle.


\section*{Solution to Exercise 23.1-8}

Consider any edge $(u, v) \in T$. If we remove $(u, v)$ from $T$, $T$ will be disconnected, resulting in a cut $(S, V - S)$. The edge $(u, v)$ is a light edge crossing the cut $(S, V - S)$ (by Exercise 23.1-3). Now consider the edge $(x, y) \in T'$ that crosses $(S, V - S)$. It, too, is a light edge crossing this cut, so that the weights of the edges $(u, v)$ and $(x, y)$ are equal. Therefore, for every edge $(u, v)$ in $T$, there must be an edge $(x, y)$ in $T'$ whose weight is equal to that of $(u, v)$, so that the list $L$ is also the sorted list of edge weights of $T'$.

Prof. Ming-Deh Huang has another solution, see \texttt{Prof Ming-Deh Huang.pdf} in directory \texttt{Unofficial Solutions to Homework}.


\section*{Solution to Exercise 23.1-9}

If there's a spanning tree $T''$ of $G'$ whose total weight is smaller than that of $T'$, we can substitute $T'$ with $T''$ in $T$ to get a spanning tree of $G$, whose total weight is smaller than that of $T$, so that $T$ is not a minimum spanning tree of $G$, which is a contradiction.


\section*{Solution to Exercise 23.1-10}

If the weight of edge $(x, y)$ is decreased by $k$, the total edge weight $W$ of $T$ will also be decreased by $k$, so that every minimum spanning tree for $G$ with edge weights given by $w'$ should have its total edge weight $W'$ not larger than $W - k$. We now show that $W'$ is not smaller than $W - k$. Every minimum spanning tree for $G$ with edge weights given by $w'$ must contain the edge $(x, y)$; otherwise because other edge weights have not changed, its total edge weight should be not smaller than $W$. For any of these minimum spanning trees, because the only edge who has its weight changed is $(x, y)$, its total edge weight must be not larger than $W - k$. Therefore, the total edge weight of any minimum spanning tree for $G$ with edge weights given by $w'$ is $W - k$, and $T$ is still such a minimum spanning tree.


\section*{Solution to Exercise 23.1-11}

Let $(u, v)$ be the edge not in $T$ whose weight is decreased. Using \proc{DFS} or \proc{BFS}, find the unique simple path from $u$ to $v$ in $T$. Find an edge $e$ of maximal weight on that path. If the weight of $e$ is greater than that of $(u, v)$, replace $e$ in $T$ with $(u, v)$.


\section*{Solution to Exercise 23.2-1}

First sort the edges in $T$, and then insert the remaining edges into the sorted list. During the insertion of each remaining edge $e$, make sure it is inserted after those edges already in the list whose weights are also $w(e)$.


\section*{Solution to Exercise 23.2-2}

The only place in \proc{MST-Prim} that makes difference between adjacency lists and adjacency matrix is line 8. If we apply an adjacency matrix implementation directly to \proc{MST-Prim}, the running time would become $O(V^2 \lg V)$.

Let's denote the elements of the adjacency matrix by $a_{ij}$.

\begin{codebox}
\Procname{$\proc{MST-Prim-Adjacency-Matrix}(G)$}
\li \For $i \gets 1$ \To $|V|$
\li   \Do
        $\id{key}[i] \gets \infty$
\li     $\pi[i] \gets \const{nil}$
\li     $\id{included}[i] \gets \const{false}$
      \End
\li $\id{key}[1] \gets 0$
\li $\id{included}[1] \gets \const{true}$
\li $n \gets 1$
\li $i \gets 1$
\li \While $n < |V|$
\li   \Do
        \For $j \gets 1$ \To $|V|$
\li       \Do
            \If $\id{included}[j] = \const{false}$ and $a_{ij} < \id{key}[j]$
\li           \Then
                $\pi[j] \gets i$
\li             $\id{key}[j] \gets a_{ij}$
              \End
          \End
\li   $k \gets 1$
\li     \For $j \gets 1$ \To $|V|$
\li       \Do
            \If $\id{included}[j] = \const{false}$ and $\id{key}[j] < \id{key}[k]$
\li           \Then
                $k \gets j$
              \End
          \End
\li     $\id{included}[k] = \const{true}$
\li     $i \gets k$
\li     $n \gets n + 1$
      \End
\end{codebox}

\end{document}
