\documentclass[a4paper, fleqn]{article}
\usepackage{clrscode}
\setlength\mathindent{0em}
\parskip = 7bp
\begin{document}

\section*{Solution to Exercise 18.2-2}

A redundant \proc{Disk-Read} happens when the root is split, and the
insertion goes down at least one level in the tree with a new root. In
this case, the two children of the new root are in memory prior to the
\proc{Disk-Read} operation in line 12 of \proc{B-Tree-Insert-Nonfull},
bacause they are parts of the former root.

There's no redundant \proc{Disk-Write}.






\section*{Solution to Exercise 18.2-3}

To find the minimum key stored in a B-tree, we can follow the leftmost
path downward from the root until we reach the leftmost leaf. The
leftmost key in the leftmost leaf is the minimum key of the tree.

To find the maximum key stored in a B-tree, we can follow the
rightmost path downward from the root until we reach the rightmost
leaf. The rightmost key in the rightmost leaf is the maximum key of
the tree.

The procedure below find the predecessor of a given key
$\id{key}_i[x]$:

\begin{codebox}
\Procname{\proc{B-Tree-Find-Predecessor}$\id{key}_i[x]$}
\If $\id{leaf}[x]$
  \Then
    \If $i > 1$
      \Then
        \Return $\id{key}_{i-1}[x]$
      \Else
        \While $x$ is not the root and $x$ is the leftmost child of its parent
          \Do
            $x \gets \mbox{ the parent of } x$
          \End
        \If $x$ is the root
          \Then
            \Return \const{nil}
          \Else
            $y \gets \mbox{ the parent of } x$
            $i \gets \mbox{ the index of child of } x { in } y$
            \Return $\id{key}_{i-1}[y]$
          \End
      \End
  \Else
    \Return the maximum key in the subtree rooted at $c_i[x]$
  \End
\end{codebox}






\section*{Solution to Problem 18-1}

\begin{enumerate}
\renewcommand{\labelenumi}{\itshape \bfseries \alph{enumi}.}

\stepcounter{enumi}  % a
\stepcounter{enumi}  % b
\stepcounter{enumi}  % c

\item  % d

Let's denote the stack as an array $S$, where $S[0]$ is the stack
bottom. We only consider the cases where we are going to push in or
pop from a page which is not in memory.

If we want to push a new element into $S[i]$, where $i \geq 1$, then
we can load the page, where $S[i]$ resides, into the memory position,
where the page containing $S[i-1]$ does \emph{not} resides, and
overwrite the page containing $S[i-1]$. If $i = 0$, then we can load
the page into either memory position.

If we want to pop the current stack head $S[i]$, then we can load the
page, where $S[i]$ resides, into the memory position, where the page
containing $S[i+1]$ resides, and overwrite the page containing
$S[i+1]$.

\end{enumerate}






\section*{Solution to Problem 18-2}

\begin{enumerate}
\renewcommand{\labelenumi}{\itshape \bfseries \alph{enumi}.}

\stepcounter{enumi}  % a

\item  % b

Suppose that $k$ is greater than the keys in $T'$, and is less than
the keys in $T''$.

If $h' = h''$, then let $k$ be the root, $T'$ be the left subtree and
$T''$ be the right subtree.

If $h' < h''$, the locate in the leftmost path of $T''$ the node $x$
of height $h'+1$. Insert $k$ as the leftmost key in $x$ and $T'$ as
the leftmost child of $x$. If $x$ contains four keys after the
insertion, then split $x$ as is done in \proc{B-Tree-Split-Child}. The
splitting operation will go upward to the root.

The situation of $h' > h''$ is symmetric to that of $h' < h''$.

\item  % c

The heights of $T'_{i-1}$ are always less than those of $T'_i$.

\item  % d

Take advantage the conclusion of \textbf{\emph{c}}. The total
asymptotic running time is $O(d_k)$, where $d_k$ is the depth of $k$.

\end{enumerate}

\end{document}
