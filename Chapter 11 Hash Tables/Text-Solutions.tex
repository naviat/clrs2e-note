\documentclass[a4paper, fleqn]{article}
\usepackage{clrscode}
\setlength\mathindent{0em}
\parskip = 7bp
\begin{document}

\section*{Solution to Exercise 11.2-3}
\begin{itemize}
\item
\textbf{\textit{Successful searches:}} As in the text, we assume that
the element being searched for is equally likely to be any of the $n$
elements stored in the table. The number of elements examined during a
successful search for an element $x$ is 1 more than the number of
elements that appear before $x$ in $x$'s list. However, in a sorted
list, elements before $x$ in the list are all smaller than $x$
(assuming that all keys are distinct), and elements after $x$ in the
list are all greater than $x$. Let $x'_i$ denote the $i$th smallest
element inserted into the table, for $i = 1, 2, \ldots, n,$ and let
$k'_i = \id{key}[x'_i]$. For keys $k'_i$ and $k'_j$, we define the
indicator random variable $X'_{ij} = \mbox{I}\{h(k'_i) =
h(k'_j)\}$. Under the assumption of simple uniform hashing, we have
Pr\{$h(k'_i) = h(k'_j)$\} = $1/m$, and so by Lemma 5.1, E[$X'_{ij}$] =
$1/m$. Thus, the expected number of elements examined in a successful
search is
\[
\mbox{E}\left[
\frac{1}{n}\sum_{i=1}^n \left(1 + \sum_{j=1}^{i-1}X'_{ij} \right)
\right],
\]
which is $\Theta(1+\alpha)$, the same as the hash table of unsorted
list version.

\item
\textbf{\textit{Unsuccessful searches:}} 

\item
\textbf{\textit{Insertions:}} The same as unsuccessful searches.

\item
\textbf{\textit{Deletions:}} The worst-case time is still $O(1)$ when
the lists are doubly linked.
\end{itemize}





\section*{Solution to Exercise 11.3-6}

Let $f(x;a_0, a_1, \ldots, a_{n-1}) = \sum_{j=0}^{n-1}a_jx^j$, then
\[
h_b(\langle a_0, a_1, \ldots, a_{n-1} \rangle) = f(b;a_0, a_1,
\ldots, a_{n-1}) \mbox{ mod } p.
\]
Suppose that $\alpha$ is an element in $U$ and $f(b;\alpha) \equiv c$
(mod $p$) where $c \in Z_p$, then $b$ is a zero of $f(x;\alpha) - c$
according to the definition in Exercise 31.4-4. Also from the
conclusion of that exercise, $f(x;\alpha) - c$ can have at most $n-1$
distinct zeros modulo $p$. Thus, there are at most $n-1$ such $b$ to
make $f(b;\alpha) \equiv c$ (mod $p$), or $h_b(\alpha) =
c$. Therefore, for any element in $U$, the number of hash functions $h
\in \mathcal{H}$ to hash this element to a specific slot is $n-1$. In
other words, with a hash function randomly chosen from $\mathcal{H}$,
the chance of a collision between two distinct elements is no more
than $(n-1) / |\mathcal{H}| = (n-1)/p$. Therefore, $\mathcal{H}$ is
$((n-1)/p)$-universal.






\section*{Solution to Problem 11-4}
\begin{enumerate}
\renewcommand{\labelenumi}{\itshape \bfseries \alph{enumi}.}
\item  % a
If the family $\mathcal{H}$ of hash functions is 2-universal, then for
each pair of distinct keys $k,l \in U$ and for any $h$ chosen at
random from $\mathcal{H}$, the pair $\langle h(k), h(l) \rangle$ is
equally likely to be any of the $m^2$ pairs with elements drawn from
${0, 1, \ldots, m-1}$. Therefore, the chance of $h(k) = h(l)$ is
$m/m^2 = 1/m$. Thus, $\mathcal{H}$ is universal.
\end{enumerate}

\end{document}
