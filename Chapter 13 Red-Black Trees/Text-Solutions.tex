\documentclass[a4paper, fleqn]{article}
\usepackage{clrscode}
\setlength\mathindent{0em}
\parskip = 7bp
\begin{document}

\section*{Solution to Exercise 13.1-7}

A red-black tree which is a complete binary tree with odd height and
all the nodes on the odd depths being red realizes the largest
possible ratio of red internal nodes to black internal nodes, which is
2. While a red-black tree with no red nodes has the smallest possible
ratio, which is zero.




\section*{Solution to Exercise 13.2-1}

\begin{codebox}
\Procname{\proc{Right-Rotate}$(T,y)$}
\li $x \gets \id{left}[y]$ \>\>\>\>\>\>\>  \Comment Set $x$.
\li $\id{left}[y] \gets \id{right}[x]$ \>\>\>\>\>\>\>
      \Comment Turn $x$'s right subtree into $y$'s left subtree.
\li \If $\id{right}[x] \neq \id{nil}[T]$
\li   \Then $p[\id{right}[x]] \gets y$
      \End
\li \If $p[x] \gets p[y]$ \>\>\>\>\>\>\>
      \Comment Link $y$'s parent to $x$.
\li \If $p[y] = \id{nil}[T]$
\li    \Then $\id{root}[T] \gets x$
\li    \Else \If $y = \id{left}[p[y]]$
\li        \Then $\id{left}[p[y]] \gets y$
\li        \Else $\id{right}[p[y] \gets y$
           \End
       \End
\li $\id{right}[x] \gets y$ \>\>\>\>\>\>\>
      \Comment Put $y$ on $x$'s right.
\li $p[y] \gets x$
\end{codebox}




\section*{Solution to Exercise 13.2-5}

An example of two trees $T_1$ and $T_2$ such that $T_1$ cannot be
right-converted to $T_2$: $T_1$ is a binary search tree that is a
right going chain; $T_2$ is a left going chain that contains the same
keys in $T_1$.

To show that if a tree $T_1$ can be right-converted to $T_2$, it can
be right-converted using $O(n^2)$ calls to \proc{Right-Rotate}, we
define the \emph{left height} of a node $y$ to be the number of edges
on the longest simple downward path from $y$ to a leaf of the
\emph{left} subtree of $y$. For any node $y$ in a binary search tree
$T$, if a call to \proc{Right-Rotate}$(T,y)$ can be made, then the
left height of $y$ must be greater than zero before the invocation;
what's more, after the invocation, the left height of $y$ will be
reduced by at least one. Since the left height of a node of an
$n$-node binary search tree is at most $n-1$, the maximum number of
calls to \proc{Right-Rotate} on a specific node is $n-1$. Therefore,
$T_1$ can be right-converted to $T_2$ using at most $n(n-1) = O(n^2)$
calls to \proc{Right-Rotate}.




\section*{Solution to Exercise 13.3-5}

Suppose that we now insert a node $z$ into a red-black tree which is
not empty.

If the body of the \kw{while} loop in \proc{RB-Insert-Fixup} is not
executed, then we will have node $z$ passed in preserved red.

Otherwise, by part (a) of the loop invariant, node $z$ keeps red in
case 1 and case 3, and its color will never be changed in the
following iterations. The only subtlety is in case 2. Because we set
$z \gets p[z]$ and do a rotation on the new $z$, whose color must be
red (otherwise the body of the \kw{while} loop won't be executed), $z$
will still be red after an iteration and its color will not be changed
in the following iterations.

Therefore, we will have at least one red node in a red-black tree
formed by inserting $n$ node with \proc{RB-Insert} and $n > 1$.




\section*{Solution to Exercise 13.3-6}

Store in an array the addresses of nodes the \kw{while} loop in
\proc{RB-Insert} has passed. Pass the array to \proc{RB-Insert-Fixup}
as the third parameter. \proc{RB-Insert-Fixup} and the rotation
procedures should also store the addresses of nodes relevant.





\section*{Solution to Exercise 13.4-1}

If the root is red before executing \proc{RB-Delete-Fixup}, then the
root must be $x$. In this case, the body of the \kw{while} loop in
\proc{RB-Delete-Fixup} won't be executed, and $x$ (the root) is set
black in line 23.

Otherwise, assume that the root is black before any iteration of the
\kw{while} loop. Notice that in each iteration of the \kw{while} loop,
only the nodes in the subtree rooted at $p[x]$ are involved to
possibly change their color or positions. Therefore, if the root of
the tree is changed to be red, it must be that the root of the subtree
becomes red in some iteration. However, consider the four cases in. In
case 1, the subtree's root is set black; while in other cases, the
subtree's root's color is unchanged. Notice that if the subtree's root
is red before an iteration, then it must not be the root of the
tree. Therefore, the root of the tree remains black after each
iteration.




\section*{Solution to Problem 13-3}
\begin{enumerate}
\renewcommand{\labelenumi}{\itshape \bfseries \alph{enumi}.}
\stepcounter{enumi}  % a
\stepcounter{enumi}  % b

\item  % c
The recursive procedure \proc{AVL-Insert}$(x,z)$ is as follows:
\begin{codebox}
\Procname{\proc{AVL-Insert}$(x,z)$}
\li \If $\id{key}[z] \leq \id{key}[x]$
\li   \Then
        \If $\id{left}[x] = \const{nil}$
\li       \Then
            $\id{left}[x] \gets z$
\li         $p[z] \gets x$
\li       \Else
            \proc{AVL-Insert}$(\id{left}[x], z)$
        \End
\li   \Else
        \If $\id{right}[x] = \const{nil}$
\li       \Then
            $\id{right}[x] \gets z$
\li         $p[z] \gets x$
\li       \Else
            \proc{AVL-Insert}$(\id{right}[x], z)$
        \End
    \End
\li \If $(\id{left}[x] \neq \const{nil} \mbox{ and } h[x] = h[\id{left}[x]])$
\li    \>$\mbox{ or } (\id{right}[x] \neq \const{nil} \mbox{ and } h[x] = h[\id{right}[x]])$
\li   \Then $h[x] \gets h[x] + 1$
    \End
\li \proc{Balance}$(x)$
\end{codebox}
\end{enumerate}

\end{document}
