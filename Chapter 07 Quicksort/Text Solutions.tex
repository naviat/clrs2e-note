\documentclass[a4paper, fleqn]{article}
\usepackage{clrscode}
\setlength\mathindent{0em}
\parskip=7bp
\begin{document}

\section*{Solution to Exercise 7.2-1}
Assume that $T(k) = \Theta(k^2)$ holds for $k \leq n-1$, that is, $c_1 (n-1)^2 
\leq T(n-1) \leq c_2 (n-1)^2$ holds for some positive constants $c_1$, $c_2$. 
Substituting the left side inequality into the recurrence yields
\begin{eqnarray*}
T(n) & \geq & c_1 (n-1)^2 + d_1 n \hspace{15mm} d_1 \mbox{ is some positive constant.} \\
     &   =  & c_1 n^2 + (d_1 - 2c_1)n + c_1 \\
     & \geq & c_1 n^2,
\end{eqnarray*}
where the last step holds as long as $c_1 \leq d_1 / 2$. Therefore, $T(n) = \Omega(n^2)$.

The $O(n^2)$ part can be similarly proved.







\section*{Solution to Exercise 7.2-6}
In this argument, we will assume that all the elements in the input array are 
distinct. For any input array of size $n$, if the $i$th smallest element is in 
position $n$, then \proc{Partition} will produce a split of $n-i$ to $i-1$, 
where $1 \leq i \leq n$. This split is more balanced than $1 - \alpha$ to 
$\alpha$ if and only if $\alpha < i/n < 1 - \alpha$. The number of such $i$ is 
approximately $(1-2\alpha)n$. Since the probability of any $i$th smallest 
element to be in a specific position is equal to $1/n$, the probability that 
\proc{Partition} produces a split more balanced than $1-\alpha$ to $\alpha$ is 
approximately $(1-2\alpha)n/n = 1-2\alpha$.







\section*{Solution to Exercise 7.3-2}
It is obvious that the number of calls made to the random-number generator 
\proc{Random} is equal to the number of calls to \proc{Randomized-Partition}. 
Therefore, this is also the number of non-leaf nodes in the recursion tree 
for \proc{Randomized-Quicksort}.

The best case occurs when each call to \proc{Randomized-Partition} partitions 
its input array into three sets --- those less than or equal to the pivot, 
those greater than the pivot, and the pivot itself. In this case, each node in 
the recursion tree corresponds to a unique element in the original input array, 
and the total number of nodes in the recursion tree is $n$, which is the size 
of the original input array. Notice that there's no degree-1 node in the 
recursion tree since each call to \proc{Randomized-Quicksort} makes either no 
or two recursive calls. Thus, letting $x$ be the number of non-leaf nodes in 
the recursion tree, we have $2x + 1 = n$ from Exercise B.5-3, and $x = (n - 1) 
/ 2$.

The worst case occurs when each call to \proc{Randomized-Partition} partitions 
its input array into only two sets --- those that are not pivot, and the pivot 
itself. In this case, there are $n$ leaves, and the number of non-leaf nodes is 
therefore $n - 1$ from Exercise B.5-3.

Thus, in either the best case or the worst case, the number of calls that are 
made to the random-number generator \proc{Random} is $\Theta(n)$.







\section*{Solution to Exercise 7.4-5}
We will adopt the method used in subsection 7.4.2, which define
\[
X_{ij} = \mbox{I}\{z_i \mbox{ is compared to } z_j\},
\]
and
\[
X = \sum_{i=1}^{n-1} \sum_{j=i+1}^n X_{ij}.
\]
Since quicksort will return when its input array has fewer than $k$ elements, 
there maybe no pivot is chosen from $Z_{ij}$ if $j - i + 1 < k$. Thus, we have
\[
\mbox{Pr}\{z_i \mbox{ is compared to } z_j\} < \frac{2}{j - i + 1}
\]
for such $Z_{ij}$, and this time we will evaluate E[$X$] using inequalities 
(A.13) and (A.14) as
\begin{eqnarray*}
\mbox{E}[X] 
& \leq & \sum_{i=1}^{n-1}\sum_{j=i+1}^n \frac{2}{j-i+1} \\
& \leq & \sum_{i=1}^{n-k+1}\sum_{j=i+k-1}^n \frac{2}{j-i+1} + 
         \sum_{i=1}^{n-k+2}\sum_{j=i+1}^{i+k-2} 1 + 
         \sum_{i=n-k+2}^{n-1}\sum_{j=i+1}^n \frac{2}{j-i+1} \\
& = & \sum_{i=1}^{n-k+1}\sum_{j=i+k-1}^n \frac{2}{j-i+1} + (n-k+1)(k-2) + 
      \sum_{i=n-k+2}^{n-1}\sum_{j=1}^{n-i+1} \frac{2}{j} \\
& \leq & \sum_{i=1}^{n-k+1}\sum_{j=i+k-1}^n \frac{2}{j-i+1} + nk + 
         \sum_{i=n-k+2}^{n-1}\sum_{j=1}^{k-1} \frac{2}{j} \\
& = & \sum_{i=1}^{n-k+1}\sum_{j=k}^{n-i+1} \frac{2}{j} + nk + 
      2(k-2)(\ln(k-1) + 1) \\
& = & \sum_{i=1}^{n-k+1}\left(\sum_{j=1}^{n-i+1}\frac{2}{j} - 
      \sum_{j=1}^{k-1}\frac{2}{j}\right) + nk + 2k(\ln k + 1) \\
& \leq & \sum_{i=1}^{n-k+1}\left(\sum_{j=1}^{n}\frac{2}{j} - 
         \sum_{j=1}^{k-1}\frac{2}{j}\right) + nk + 2k(\ln k + 1) \\
& \leq & 2\sum_{i=1}^{n-k+1}\left(\ln n + 1 - \ln k\right) + nk + 2nk \\
& = & 2(n-k+1)(\ln(n/k) + 1) + nk + 2nk \\
& \leq & 2n(\ln(n/k) + 1) + 3nk \\
& = & O(n\lg(n/k) + nk).
\end{eqnarray*}
\indent
Now consider the insertion sort part. If the expected number of partitions 
resulting from the quicksort part is $m$, then each partition will have $n/m$ 
elements on average, and the insertion sort part will run in $O(m(n/m)^2) = 
O(n^2/m)$ expected time. We know that $m > n/k$ since each partition contains 
fewer than $k$ elements. Therefore, the insertion sort part runs in $O(n^2/(n/k)) 
= O(nk)$ expected time. The whole sorting algorithm runs in $O(nk + n\lg(n/k))$ 
expected time.







\section*{Solution to Exercise 7.4-6}
Again, we assume that all elements in the input array are distinct. In this case, 
the probability that the $i$th smallest element becomes the pivot is
\begin{eqnarray*}
\Pr\{z_i \mbox{ is the pivot}\} & = & p_i \\
& = & \frac{(i - 1)(n - i)}{\left(\!\begin{array}{c}n \\ 3\end{array}\!\right)} \\
& = & \frac{6(-i^2 + (n + 1)i - n)}{n(n - 1)(n - 2)}.
\end{eqnarray*}
where $2 \leq i \leq n - 1$ and $n \geq 3$.

Supposing that $0 < \alpha \leq 1/2$, the probability of getting at worst an 
$\alpha$-to-$(1 - \alpha)$ split is approximately
\begin{eqnarray*}
P & = & \sum_{i = \alpha n}^{(1 - \alpha)n} p_i \\
  & = & \frac{6}{n(n - 1)(n - 2)} \sum_{i = \alpha n}^{(1 - \alpha)n} (-i^2 + (n + 1)i - n) \\
  & = & \frac{6}{n(n - 1)(n - 2)} \left(\frac{(1 - 2\alpha)n^2(n+1)}{2} - (1 - 2\alpha)n^2 - n - \sum_{i = \alpha n}^{(1 - \alpha)n} i^2 \right) \\
  & = & \frac{6}{n(n - 1)(n - 2)} \left(\frac{(1 - 2\alpha)n^2(n-1)}{2} - n - \sum_{i=1}^{(1-\alpha)n} i^2 + \sum_{i=1}^{\alpha n - 1} i^2 \right) \\
  & = & \frac{6}{n(n - 1)(n - 2)} \left(\frac{(1 - 2\alpha)n^2(n-1)}{2} - n - \right. \\
  &   & \left. \frac{(1-\alpha)n((1-\alpha)n + 1)(2n(1-\alpha) + 1)}{6} + \frac{(\alpha n - 1)\alpha n(2\alpha n - 1)}{6} \right) \\
  & = & \frac{6}{n(n - 1)(n - 2)} \left( \frac{(4\alpha^3 - 6\alpha^2 + 1)n^3 - 6(\alpha - 1)^2 n^2 + (2\alpha - 1)n}{6} - n \right) \\
  & = & \frac{(4\alpha^3 - 6\alpha^2 + 1)n^2 - 6(\alpha - 1)^2 n + 2\alpha - 7}{(n - 1)(n - 2)} \\
  & = & 4\alpha^3 - 6\alpha^2 + 1.
\end{eqnarray*}
The last step holds when $n \rightarrow \infty$.







\section*{Solution to Problem 7-1}
\begin{enumerate}
\renewcommand{\labelenumi}{\itshape \bfseries \alph{enumi}.}

\stepcounter{enumi}

\item  % b
We will prove it by induction. As an initialization, notice that $i$ and $j$ 
never result in values outside $p \ldots r$ after the first iteration of the 
\kw{while} loop because $A[p] = x$ so that $A[p]$ stands as a sentinel for $i$ 
and $j$. For the $k$th iteration where $k \geq 2$, we will have at the beginning 
$i < j$, $A[i] \leq x$ and $A[j] \geq x$ resulting from the previous iteration, 
so that $A[i]$ stands as a sentinel for $j$, and $A[j]$ stands as a sentinel for 
$i$, respectively.




\item  % c
In \proc{Hoare-Partition}, $j$ keeps decreasing in the \kw{while} loop of lines 
4--11 with its initial value to be $r + 1$. Therefore, \proc{Hoare-Partition} 
will terminate with $j = r$ only if the \kw{while} loop of lines 4--11 iterates 
exactly once. But notice that $A[p] = x$ and $i = p - 1$ before the first 
iteration, so $i$ will always stop at value $p$ in the first iteration of the 
\kw{while} loop. Thus, in this case, we will result in $i = p$ and $j = r$ in 
the first iteration of the \kw{while} loop, so that $i < j$, and the \kw{while} 
loop will have its second iteration due to the \kw{if} statements of lines 
9--11. This is contratictory to the assumption that the \kw{while} loop iterates 
exactly once. Thus, it is impossible that \proc{Hoare-Partition} terminates with 
$j = r$. According to \textit{\textbf{b}}, \proc{Hoare-Partition} will only 
return a value $j$ such that $p \leq j < r$.




\item  % d
To show that every element of $A[p \twodots j]$ is less than or equal to every 
element of $A[j+1 \twodots r]$, we use the following loop invariant:
\begin{quote}
Before each iteration of the \kw{while} loop of lines 4--11, every element of 
$A[p \twodots i-1]$ is less than or equal to $x$ , while every element of 
$A[j+1 \twodots r]$ is greater than or equal to $x$.
\end{quote}
We need to show that this invariant is true prior to the first loop iteration, 
that each iteration of the loop maintains the invariant, and that the invariant 
provides a useful property to show correctness when the loop terminates.
\begin{description}

\item[Initialization:]
Prior to the first loop iteration, $i = p - 1$ and $j = r + 1$. Thus, $A[p \twodots i-1]$ 
and $A[j+1 \twodots r]$ are empty, and the loop invariant is trivially satisfied.

\item[Maintenance:]
For the first iteration, $i$ stops at value $p$ because $A[p] = x$, so that 
$A[p \twodots i-1]$ is still empty; while $j$ stops decreasing at the first 
element it meets which is less than or equal to $x$, so every element in $A[j+1 \twodots r]$ 
is greater than $x$.

For the $k$th iteration where $k \geq 2$, we have at the beginning $i < j$, 
$A[i] \leq x$ and $A[j] \geq x$ resulting from the previous iteration. During 
the $k$th iteration, $i$ stops increasing at the first element it meets which 
is greater than or equal to $x$, while $j$ stops decreasing at the first element 
it meets which is less than or equal to $x$. Thus, the loop invariant still 
holds before the next iteration.

\item[Termination:]
At the termination, we have $j = i$ or $j = i - 1$ because $A[j+1] \geq x$, and 
the \kw{while} loop won't stop until $i \geq j$. Note that at the termination, 
we have $A[j] \leq x$ and $A[i] \geq x$ because the exchanging operation isn't 
executed in the last iteration).

If $j = i$, it can be inferred that $A[j] = x$, then every element in $A[p \twodots j]$ 
is less than or equal to $x$, and every element in $A[j+1 \twodots r]$ is 
greater than or equal to $x$ according to the loop invariant. If $j = i - 1$, 
it also holds trivially. Therefore, every element of $A[p \twodots j]$ is less 
than or equal to every element of $A[j+1 \twodots r]$ when \proc{Hoare-Partition} 
terminates.

\end{description}




\item  % e
Here's the rewritten \proc{Quicksort} procedure using \proc{Hoare-Partition}:
\begin{codebox}
\Procname{\proc{Quicksort}$(A,p,r)$}
\li \If $p < r$
\li     \Then
            $q \gets \proc{Hoare-Partition}(A,p,r)$
\li         \proc{Quicksort}$(A,p,q)$
\li         \proc{Quicksort}$(A,q+1,r)$
        \End
\end{codebox}

\end{enumerate}







\section*{Solution to Problem 7-2}
\begin{enumerate}
\renewcommand{\labelenumi}{\itshape \bfseries \alph{enumi}.}

\item  % a
Pr\{$A[i]$ is chosen as the pivot where $p \leq i \leq r$\} $= 1 / (r - p + 1) = 
1 / n$ because a call to \proc{Random}$(p,r)$ in line 2 in \proc{Randomized-Partition} 
returns an integer between $p$ and $r$, inclusive, with each such integer being 
equally likely, while $r - p + 1 = n$. Assuming that all elements are distinct, 
we have E[$X_i$] = Pr\{$X_i = 1$\} = $1 / n$.




\item  % b
This is because
\[
T(n) = \sum_{q=1}^n X_q(T(q - 1) + T(n - q) + \Theta(n)),
\]
Note that there is exactly one $X_q = 1$.




\item  % c
From equation (7.5), we have
\begin{eqnarray*}
\mbox{E}[T(n)]
& = & \sum_{q = 1}^n \mbox{E}[X_q (T(q - 1) + T(n - q) + \Theta(n))] \\
& = & \sum_{q = 1}^n (\mbox{E}[X_q T(q - 1)] + \mbox{E}[X_q T(n - q)] + \Theta(n) \mbox{E}[X_q]) \\
& = & \sum_{q = 1}^n \mbox{E}[X_q T(q - 1)] + \sum_{q - 1}^n \mbox{E}[X_q T(n - q)] + \Theta(n) \sum_{q = 1}^n \mbox{E}[X_q] \\
& = & 2\sum_{q = 1}^n \mbox{E}[X_q T(q - 1)] + \Theta(n).
\end{eqnarray*}
Since the event of the $q$th smallest element being chosen as the pivot is 
independent of how many time it takes to sort the $q - 1$ smallest elements in 
the input array, we can continue the above equations as
\begin{eqnarray*}
\mbox{E}[T(n)]
& = & 2\sum_{q = 1}^n \mbox{E}[X_q] \mbox{E}[T(q - 1)] + \Theta(n) \\
& = & \frac{2}{n}\sum_{q = 1}^n \mbox{E}[T(q - 1)] + \Theta(n).
\end{eqnarray*}




\item  % d
\begin{eqnarray*}
\sum_{k=2}^{n-1} k \lg k
& = & \sum_{k=2}^{\lceil n/2 \rceil - 1} k \lg k + \sum_{k=\lceil n/2 \rceil}^{n-1} k \lg k \\
& \leq & \sum_{k=2}^{\lceil n/2 \rceil - 1} k \lg(n/2) + \sum_{k=\lceil n/2 \rceil}^{n-1} k \lg n \\
& = & \sum_{k=2}^{\lceil n/2 \rceil - 1} k (\lg n - 1) + \sum_{k=\lceil n/2 \rceil}^{n-1} k \lg n \\
& = & \sum_{k=2}^{n-1} k \lg n - \sum_{k=2}^{\lceil n/2 \rceil - 1} k \\
& = & \frac{(n-2)(n+1)}{2} \cdot \lg n - \frac{(\lceil n/2 \rceil - 2)(\lceil n/2 \rceil + 1)}{2} \\
& \leq & \frac{(n-2)(n+1)}{2} \cdot \lg n - \frac{(n/2-2)(n/2+1)}{2} \\
& = & \frac{1}{2} n^2 \lg n - \left(\frac{n}{2} + 1\right) \lg n - \frac{n^2}{8} + \left(\frac{n}{4} + 1\right) \\
& \leq & \frac{1}{2} n^2 \lg n - \frac{n^2}{8}.
\end{eqnarray*}
The last step holds because $\lg n \geq 1$ for $n \leq 2$ and $n/2 + 1 > n/4 + 1$.



\item  % e
Suppose that $\mbox{E}[T(k)] \leq ak \lg k$ holds for all $2 \leq k \leq n - 1$ 
and some positive constant $a$. Substituting into the recurrence (7.6) yields
\begin{eqnarray*}
\mbox{E}[T(n)]
& \leq & \frac{2}{n} \sum_{k=2}^{n-1} ak \lg k + \Theta(n) \\
& \leq & \frac{2}{n} \left(\frac{a}{2} n^2 \lg n - \frac{a}{8}n^2\right) + \Theta(n) \\
& = & an \lg n - \frac{a}{4}n + \Theta(n) \\
& \leq & an \lg n,
\end{eqnarray*}
since we can pick the $a$ large enough so that the quantity $-an/4$ 
dominates the $\Theta(n)$ term. The second step holds according to equation 
(7.6). Thus, E$[T(n)] = O(n \lg n)$.

Before proving that E$[T(n)] = \Omega(n \lg n)$, first see that
\[
\sum_{k = 1}^{n} k \lg k \geq \sum_{k = 1}^{n} k \lg (n / e).
\]
To prove this to be true, let's see
\begin{eqnarray*}
\sum_{k = 1}^{n} k \lg k - \sum_{k = 1}^{n} k \lg (n / e)
&   =  & \sum_{k = 1}^n k \lg (ek / n) \\
& \geq & \sum_{k = 1}^n \lg (ek / n) \\
&   =  & \lg \left(\frac{e^n n!}{n^n}\right).
\end{eqnarray*}
According to Stirling's approximation, we know that $n! \geq (n / e)^n$. 
Therefore, $e^n n! / n^n \geq 1$ so that $\lg (e^n n! / n^n) \geq 0$, that is,
\[
\sum_{k = 1}^{n} k \lg k - \sum_{k = 1}^{n} k \lg (n / e) \geq 0.
\]
Thus, we have
\begin{eqnarray*}
\sum_{k = 2}^{n - 1} k \lg k
&   =  & \sum_{k = 1}^{n} k \lg k - n \lg n \\
& \geq & \sum_{k = 1}^{n} k (\lg n - \lg e) - n \lg n \\
&   =  & \frac{1}{2} (n^2 + n)(\lg n - \lg e) - n \lg n \\
&   =  & \frac{1}{2} n^2 \lg n - \frac{\lg e}{2} n^2 - \frac{1}{2} n \lg n - \frac{\lg e}{2} n.
\end{eqnarray*}
Now suppose that $\mbox{E}[T(k)] \geq bk \lg k$ holds for all $2 \leq k \leq n - 1$ 
and some positive constant $b$. Substituting into the recurrence in equation 
(7.6) yields
\begin{eqnarray*}
\mbox{E}[T(n)]
& \geq & \frac{2}{n} \sum_{k = 2}^{n - 1} bk \lg k + \Theta(n) \\
& \geq & \frac{2b}{n} \cdot \left(\frac{1}{2} n^2 \lg n - \frac{\lg e}{2} n^2 - \frac{1}{2} n \lg n - \frac{\lg e}{2} n\right) + \Theta(n) \\
&   =  & bn \lg n - bn\lg e - b \lg n - b \lg e + \Theta(n) \\
& \geq & bn \lg n,
\end{eqnarray*}
since we can pick the constant $b$ small enough so that the $\Theta(n)$ term 
dominates the quantity $bn\lg e + b \lg n + b \lg e$. Therefore, we have 
E$[T(n)] = \Omega(n \lg n)$, so that E$[T(n)] = \Theta(n \lg n)$.

\end{enumerate}







\section*{Solution to Problem 7-3}

\begin{enumerate}
\renewcommand{\labelenumi}{\itshape \bfseries \alph{enumi}.}

\item  % a

We will argue it by induction. As the base cases, trivially, \proc{Stooge-Sort} 
correctly sort the input array $A[1 \twodots n]$ if $n = 1$ or $n = 2$. As an 
induction, suppose that the three recursive calls in lines 6--8 correctly sort 
their input arrays. After the recursive call in line 6, the one-third largest 
elements are guaranteed to be in the last two-thirds of array $A[i \twodots j]$, 
so that after the next recursive call in line 7, they are placed in the last 
one-third positions of array $A[i \twodots j]$ in sorted order. Therefore, after 
the last recursive call in line 8, the smallest two-thirds elements are placed 
in the first two-thirds positions in sorted order, so that the whole array 
$A[i \twodots j]$ is sorted.




\item  % b

The recurrence for the worst-case running time is
\[
T(n) = 3T(\lceil 2n / 3 \rceil) + \Theta(1).
\]
According to the master theorem, the asymptotic bound on the recurrence is
\[
T(n) = \Theta(n^{\log_{3/2}3}).
\]




\item  % c

Of course not, its worst-case running time is even longer than that of insersion 
sort!

\end{enumerate}







\section*{Solution to Problem 7-5}
\begin{enumerate}
\renewcommand{\labelenumi}{\itshape \bfseries \alph{enumi}.}

\item  % a
\begin{eqnarray*}
p_i
& = & \frac{(i - 1)(n - i)}{\left(\!\begin{array}{c}n \\ 3\end{array}\!\right)} \\
& = & \frac{6(i - 1)(n - i)}{n(n - 1)(n - 2)}.
\end{eqnarray*}



\item  % b
\begin{eqnarray*}
& & \lim_{n \rightarrow \infty} (p_{\lfloor (n + 1) / 2 \rfloor} - 1 / n) \\
& = & \lim_{n \rightarrow \infty} \left( \frac{6(\lfloor (n + 1) / 2 \rfloor - 1)(n - \lfloor (n + 1) / 2 \rfloor)}{n(n - 1)(n - 2)} - \frac{1}{n} \right) \\
& = & \left(\frac{6(1/2 \cdot (1 - 1/2))}{1 \cdot 2}\right)\lim_{n \rightarrow \infty} \frac{1}{n} \\
& = & 0.
\end{eqnarray*}



\item  % c
\begin{eqnarray*}
& & \sum_{i = n / 3}^{2n / 3} p_i - 1 / 3 \\
& = & \sum_{i = n / 3}^{2n / 3} \frac{6(i - 1)(n - i)}{n(n - 1)(n - 2)} - \frac{1}{3} \\
& = & \frac{6}{n(n - 1)(n - 2)} \sum_{i = n / 3}^{2n / 3}(-i^2 + (n + 1)i - n) - \frac{1}{3} \\
& = & \frac{6}{n(n - 1)(n - 2)} \left(\sum_{i = n / 3}^{(n + 1) / 2 - 1}(-i^2 + (n + 1)i - n) +\right. \\
&   & \left.\sum_{i = (n + 1) / 2 + 1}^{2n / 3}(-i^2 + (n + 1)i - n)\right) - \frac{1}{3} \hspace{1cm} (n \rightarrow \infty) \\
& \leq & \frac{6}{n(n - 1)(n - 2)} \left(\int_{i = n / 3}^{(n + 1) / 2}(-i^2 + (n + 1)i - n) + \right. \\
&      & \left.\int_{i = (n + 1) / 2}^{2n / 3}(-i^2 + (n + 1)i - n)\right) - \frac{1}{3} \\
& = & \frac{6}{n(n - 1)(n - 2)} \int_{i = n / 3}^{2n / 3} (-i^2 + (n + 1)i - n) - \frac{1}{3} \\
& = & \frac{6(-i^3 / 3 + (n + 1)i^2 / 2 - ni)|_{n / 3}^{2n / 3}}{n(n - 1)(n - 2)} - \frac{1}{3} \\
& = & 6 \cdot \left(-\frac{1}{3} \cdot \left(\left(\frac{2}{3}\right)^3 - \left(\frac{1}{3}\right)^3\right) + \frac{1}{2} \cdot \left(\left(\frac{2}{3}\right)^2 - \left(\frac{1}{3}\right)^2\right)\right) - \frac{1}{3} \\
&    & (n \rightarrow \infty) \\
& = & 6 \cdot \left(-\frac{7}{81} + \frac{3}{18}\right) - \frac{1}{3} \\
& = & \frac{13}{27} - \frac{1}{3} \\
& = & \frac{4}{27}.
\end{eqnarray*}



\item  % d
The best case occurs when each call to the partitioning procedure produces two 
subproblems with each of size no more than $n/2$. This also applies to the 
median-of-3 method. The median-of-3 method takes additional time to pick two 
more elements from the input array and to determine the median of the 3 picked 
elements. However, that additional time is only $\Theta(1)$.

\end{enumerate}







\section*{Solution to Problem 7-6}
\begin{enumerate}
\renewcommand{\labelenumi}{\itshape \bfseries \alph{enumi}.}

\item  % a
To take advantage of overlapping intervals, notice that we don't have to sort
the intervals strictly on $a_i$. If two intervals $i_j$ and $i_k$ overlap and 
$i_j$ appears earlier in order, there always exist $c_j, c_k \in i_j \cap i_k$ 
satisfying $c_j \leq c_k$ where $c_k \leq c_j$. Note that in the following 
implementations, $A$ is the input array of intervals, where $A[j] = i_j = 
[a_j, b_j]$.

The \proc{Fuzzy-Partition} procedure partitions the input array $A$ into two 
parts. At the termination, all the intervals in $A[j \twodots r]$ overlap, i.e., 
there exists a value $x$ such that $x \in [a_k,b_k]$ for all $j \leq k \leq r$. 
The left endpoint of the overlapping region of $A[j \twodots r]$ is also 
returned.
\begin{codebox}
\Procname{\proc{Fuzzy-Partition}$(A, p, r)$}
\li $j \gets r$
\li $\id{max-a} \gets a_r$
\li $\id{min-b} \gets b_r$
\li \For $k \gets r - 1$ \Downto $p$
\li     \Do \Comment Examine the intervals that overlap together with $A[r]$
\li         \If $a_k \leq \id{min-b}$ and $b_k \geq \id{max-a}$
\li             \Then
                    $\id{max-a} \gets \max(\id{max-a}, a_k)$
\li                 $\id{min-b} \gets \min(\id{min-b}, b_k)$
\li                 $j \gets j - 1$
\li                 exchange $A[k] \leftrightarrow A[j]$
                \End
        \End
\li \Return $(j,\id{max-a})$            
\end{codebox}

The \proc{Partition} procedure is slightly different from that in page 146. This 
one takes its fourth argument as the pivot, and partitions the intervals in 
$A[p \twodots r]$ into two parts. At the termination, the left endpoints of the 
intervals in $A[p \twodots i]$ are all less than or equal to the pivot, while 
the left endpoints of the intervals in $A[i+1 \twodots r]$ are all greater than 
the pivot.
\begin{codebox}
\Procname{\proc{Partition}$(A,p,r,a_{pivot})$}
\li $i \gets p - 1$
\li \For $j \gets p$ \To $r$
\li     \Do
            \If $a_j \leq a_{pivot}$
\li             \Then
                    $i \gets i + 1$
\li                 exchange $A[i] \leftrightarrow A[j]$
                \End
        \End
\li \Return $i + 1$
\end{codebox}

Here comes the sorting algorithm:
\begin{codebox}
\Procname{\proc{Fuzzy-Sort}$(A,p,r)$}
\li \If $p < r$
\li     \Then
            $(s,\id{max-a}) \gets \proc{Fuzzy-Partition}(A,p,r)$
                                                    \label{li:fuzzy-sort-n-begin}
\li         $t \gets \proc{Partition}(A,p,s-1,\id{max-a})$
                                                    \label{li:fuzzy-sort-partition}
\li         \If $t < s$
\li             \Then
                    \If $r - s + 1 > s - t$
\li                     \Then
                            \For $i \gets t$ \To $s - 1$
\li                             \Do exchange $A[i] \leftrightarrow A[r - s + i + 1]$
                                \End
\li                     \Else
                            \For $i \gets s$ \To $r$
\li                             \Do exchange $A[i] \leftrightarrow A[t + i - s]$
                                \End
                        \End
                \End
                                                    \label{li:fuzzy-sort-n-end}
\li         \proc{Fuzzy-Sort}$(A,p,t-1)$
\li         \proc{Fuzzy-Sort}$(A,r-s+t+1,r)$
        \End
\end{codebox}



\item  % b
Lines \ref{li:fuzzy-sort-n-begin}--\ref{li:fuzzy-sort-n-end} takes $\Theta(n)$ 
running time. If $a_i$ and $b_i$ are taken randomly from real set, then the 
probability that any two intervals overlap are quite small, closed to zero 
(which is equal to the probability that two integers randomly and independently 
taken from the integer set are equal). Therefore, we can apply the analysis in 
Problem 7-2 to the expected running time of \proc{Fuzzy-Sort}, which is 
consequently $\Theta(n \lg n)$.

When all of the intervals overlap, \proc{Fuzzy-Sort} won't make any recursive 
call because after the execution of lines \ref{li:fuzzy-sort-n-begin}--\ref{li:fuzzy-sort-partition} 
in \proc{Fuzzy-Sort}, we have $t = s = p$. In this case, the running time of 
\proc{Fuzzy-Sort} is $\Theta(n)$.

\end{enumerate}

\end{document}

