\documentclass[a4paper, fleqn]{article}
\usepackage{clrscode}
\setlength\mathindent{0em}
\parskip=7bp
\begin{document}

\section*{Solution to Exercise 8.1-2}

$\lg(n!) = O(n \lg n)$ because $\lg(n!) \leq \lg(n^n) = n \lg n$, while 
$\lg(n!) = \Theta(n \lg n)$ because
\begin{eqnarray*}
\sum_{k=1}^{n} \lg k
& \geq & \sum_{k=\lceil n / 2 \rceil}^n \lg k \\
& \geq & \sum_{k=\lceil n / 2 \rceil}^n \lg(n/2) \\
&  =   & (n - \lceil n / 2 \rceil + 1)\lg(n/2) \\
& \geq & (n - n / 2)\lg(n/2) \\
&  =   & (n/2)\lg(n/2) \\
&  =   & \frac{1}{2} n \lg n - \frac{n}{2}.
\end{eqnarray*}







\section*{Solution to Exercise 8.2-2}

Suppose that we have $A[s] = A[t]$ and $s < t$ in the input array. Since $j$ 
goes down from \id{length}$[A]$ to 1 in the \kw{for} loop of lines 9--11, $j$ 
will meet $t$ first and put $A[t]$ in $C[A[t]]$. After decreasing $C[A[t]]$, 
$j$ will meet $s$ and put $A[s]$ in $C[A[s]] = C[A[t]]$, so $A[s]$ will be put 
at a lower index position than that of $A[t]$ in $B$, i.e., $A[s]$ and $A[t]$ 
still appear in the same order as they do in the input array.







\section*{Solution to Exercise 8.2-4}

\begin{codebox}
\Procname{\proc{Preprocess}$(A,k)$}
\li \For $i \gets 0$ \To $k$
\li     \Do
            $C[i] \gets 0$
        \End
\li \For $j \gets 1$ \To \id{length}$[A]$
\li     \Do
            $C[A[j]] \gets C[A[j] + 1$
        \End
\li \For $i \gets 1$ \To $k$
\li     \Do
            $C[i] \gets C[i] + C[i - 1]$
        \End
\li \Return $C$
\end{codebox}

\begin{codebox}
\Procname{\proc{Counting-Query}$(C,a,b)$}
\li \If $a = 0$
\li     \Then \Return $C[b]$
\li     \Else \Return $C[b] - C[a - 1]$
        \End
\end{codebox}







\section*{Solution to Exercise 8.3-2}

Another scheme that makes any sorting algorithm stable is to make all the 
elements unique in the input array before sorting them, and change them back to 
the original values after sorting:
\begin{codebox}
\Procname{\proc{Stable-Sort}$(A)$}
\li $n \gets \id{length}[A]$
\li $m \gets \lceil \lg(n + 1) \rceil$
\li \For $i \gets 1$ \To $n$                           \label{li:stable-sort-for-1-begin}
\li     \Do
            $A[i] \gets A[i] * 2^m + i$                \label{li:stable-sort-for-1}
        \End
\li sort array $A$ by any sorting algorithm
\li \For $i \gets 1$ \To $n$
\li     \Do
            $A[i] \gets \lfloor A[i] / 2^m \rfloor$    \label{li:stable-sort-for-2}
        \End
\end{codebox}
After the \kw{for} loop of lines \ref{li:stable-sort-for-1-begin}--\ref{li:stable-sort-for-1}, 
each element in array $A$ contains two parts, with the higher order bits storing 
the original value, and the lower order bits storing the index. Therefore, if 
two elements in the original array $A$ are different, they are sorted as expected 
trivially; if they are originally equal, their ties are broken by their indices.

\proc{Stable-Sort} takes $\Theta(n)$ additional time. For $n$ elements, their 
indices are $1 \ldots n$. Each can be written in $m = \lceil \lg(n + 1) \rceil$ 
bits, so together they take $\Theta(n \lg n)$ additional space.

Note that the multiplying operation in line \ref{li:stable-sort-for-1} and the 
dividing operation in line \ref{li:stable-sort-for-2} can be replaced by 
bit-moving operation in many programming languages.







\section*{Solution to Exercise 8.3-5}
\begin{enumerate}
\item
$d$ passes in the worst case.
\item
In the worst case, 9 of 10 piles of cards in the sorting on the $(i + 1)$th 
digit are needed to keep track of when sorting one piles of cards on the $i$th 
digit. Therefore, the number of piles of cards needed to keep track of when 
sorting one piles of cards on the 1st digit is $9(d - 1)$ in the worst case, 
this is also the maximum number of piles of cards needed to keep track of in the 
worst case.
\end{enumerate}







\section*{Solution to Problem 8-5}
\begin{enumerate}
\renewcommand{\labelenumi}{\itshape \bfseries \alph{enumi}.}
\stepcounter{enumi}
\stepcounter{enumi}
\stepcounter{enumi}

\item  % d

Use heapsort to sort the subarray of $A[i], A[i + k], \ldots, A[n - k + i]$ 
where $i = 1, 2, \ldots, k-1$ or $k$. This takes $O(n / k \cdot \lg(n / k))$ 
running time. There are $k$ such subarrays. Thus, the whole sorting algorithm 
takes $O(n \lg(n / k))$ running time.


\stepcounter{enumi}


\item  % f

A way to sort an $n$-element array is to $k$-sort the array first, then sort the 
resulted $k$-sorted array. In this case, let us denote the time to sort an 
$n$-element array by $T(n)$, the time to $k$-sort an $n$-element array by $T_1(n)$, 
and the time to sort an $k$-sorted array of length $n$ by $T_2(n)$. Then we have 
$T(n) = T_1(n) + T_2(n)$ such that the time to $k$-sort an $n$-element array is 
$T_1(n) = T(n) - T_2(n)$. From the lower bound on comparison sorts, we have $T(n) 
= \Omega(n \lg n)$ in the worst case. From part \textit{\textbf{e}}, we have 
$T_2(n) = O(n \lg k) = O(n)$ since $k$ is a constant. Therefore, $T_1(n) = 
\Omega(n \lg n) - O(n) = \Omega(n \lg n)$ in the worst case.

\end{enumerate}







\section*{Solution to Problem 8-6}
\begin{enumerate}
\renewcommand{\labelenumi}{\itshape \bfseries \alph{enumi}.}

\stepcounter{enumi}

\item  % b

According to \textit{\textbf{a}}, there are at least 
$\left(\!\!\begin{array}{c}2n \\ n\end{array}\!\!\right)$ 
leaves in the decision tree. Thus, the height of the tree is
\begin{eqnarray*}
h 
& \geq & \lg\left(\!\!\begin{array}{c}2n \\ n\end{array}\!\!\right) \\
& = & \lg\left(\frac{(2n)!}{(n!)^2}\right) \\
& = & \lg((2n)!) - 2\lg(n!) \\
& \geq & \lg\left(\left(\frac{2n}{e}\right)^{2n}\right) - 2\lg(n!) \\
& = & 2n\lg(2n) - 2n \lg e - 2\lg(n!) \\
& = & 2n \lg n + 2n - 2n \lg e - 2\lg(n!).
\end{eqnarray*}
According to equation (3.19), we have
\begin{eqnarray*}
\lg(n!)
& \leq & \lg\left(\sqrt{2 \pi n} \left(\frac{n}{e}\right)^n e\right) \\
& = & \frac{1}{2}\lg(2 \pi n) + n \lg\left(\frac{n}{e}\right) + \lg e\\
& = & \frac{1}{2} \lg n + \frac{1}{2}\lg(2\pi) + n \lg n - n \lg e + \lg e\\
& = & n \lg n - n \lg e + \frac{1}{2} \lg n + \lg(\sqrt{2\pi}e).
\end{eqnarray*}
Now we can continue the bounding on $h$ as
\begin{eqnarray*}
h
& \geq &  2n \lg n + 2n - 2n \lg e - 
          2\cdot\left(n \lg n - n \lg e + \frac{1}{2} \lg n + \lg(\sqrt{2\pi}e)\right) \\
& = & 2n - \left(\lg n + \lg(2 \pi e^2)\right) \\
& = & 2n - o(n).
\end{eqnarray*}
Thus, any algorithm merging two sorted lists uses at least $2n - o(n)$ comparisons 
in the worst case.



\item  % c

If we view the $n$ input elements as a network with $n$ vertices and each vertex 
being connected to other $n - 1$ vertices by undirected edges, then a comparison 
sort can be modeled as checking the edges in the network to select $n - 1$ edges 
so that the $n$ vertices form a sorted sequence. Therefore, in a comparison sort, 
each edge that connects two elements consecutive in the sorted order must be 
checked. That is, if two elements are consecutive in the sorted order, then they 
must be compared.

The procedure to merge two sorted lists by comparisons can also be viewd as a 
comparison sort. However, in this case, there are $2(n - 1)$ edges having been 
checked at the beginning, in which either $n - 1$ edges form a previously sorted 
list of length $n$. Therefore, if two elements are consecutive in the sorted order 
and from opposite lists, then the edge connecting them have not been checked at 
the beginning, and it must be checked in the merging procedure, i.e., the two 
elements must be compared.



\item  % d

Let $L[1 \twodots n]$ and $R[1 \twodots n]$ be the two sorted lists to be merged. 
According to part \textit{\textbf{c}}, in the case that $L[1] < R[1] < L[2] < R[2] 
< \cdots < L[n - 1] < R[n - 1] < L[n] < R[n]$, $L[1]$ must be compared to $R[1]$, 
$R[1]$ must be compared to $L[2]$, \ldots, and so on. Therefore, there are 
$2n - 1$ comparisons. Thus, the lower bound on the worst-case number of comparisons 
is $2n - 1$.

\end{enumerate}

\end{document}

