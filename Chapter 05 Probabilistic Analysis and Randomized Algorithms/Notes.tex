\documentclass[a4paper, fleqn]{article}

\usepackage{clrscode}

\setlength\mathindent{0em}
\parskip = 7bp

\begin{document}

\section{\proc{Randomize-In-Place} is similar to selection sort}

\proc{Randomize-In-Place} behaves quite the same as the \textbf{\textit{selection 
sort}} algorithm described in Exercise 2.2-2, except that it randomly chooses one 
element but not the smallest element in $A[i \twodots n]$ and exchange it with 
$A[i]$. Therefore, for any particular permutation $\langle\ x_1, x_2, \ldots, x_n 
\rangle$, $x_i$ is put in position $i$ with probability $1 / (n - i + 1)$, and 
the permutation occurs with probability $1/n!$.


\section{Is it rigorous of the proof of the lower bound of streak?}

In Subsection 5.4.3, the textbook gives the probability that every one of the 
$\lfloor (\lg n) / 2 \rfloor$ groups fails to be a streak of length $\lfloor 
(\lg n) / 2 \rfloor$ (let's denote it by $P_1$ on page 112, and uses it to bound 
the probability that the longest streak exceeds $\lfloor (\lg n) / 2 \rfloor$.

However, $P_1$ is not the probability that the longest streak of consecutive 
heads is less than $\lfloor (\lg n) / 2 \rfloor$ (let's denote this probability 
by $P_2$)! Imagine that two neighboring group both fail to be a streak of length 
$\lfloor (\lg n) / 2 \rfloor$, but the former group has a streak of $t$ in its 
final tosses, where $1 < t < \lfloor (\lg n) / 2 \rfloor$, and the later group 
has a streak of $\lfloor (\lg n) / 2 \rfloor - t + 1$ in its initial tosses, 
then there will be a sreak that is longer than $\lfloor (\lg n) / 2 \rfloor$. 
Therefore, the events corresponding to $P_2$ is more ``strict'', i.e. they are 
subsets of the events that correspond to $P_1$, and $P_2 \leq P_1$. Thus, the 
probability that the longest streak is \textit{at least} $\lfloor (\lg n) / 2 
\rfloor$ is
\begin{eqnarray*}
\sum_{j = \lfloor (\lg n) / 2 \rfloor}^n \mbox{Pr}\{L_j\} & = & 1 - P_2 \\
& \geq & 1 - P_1 \\
& = & 1 - O(1/n).
\end{eqnarray*}
So, there's no big problem in the proof of the lower bound. But the word 
``exceeds'' in line 3 on page 113 should be changed to ``at least'', and so is 
the corresponding computing procedure of E$[L]$ below.

\end{document}
