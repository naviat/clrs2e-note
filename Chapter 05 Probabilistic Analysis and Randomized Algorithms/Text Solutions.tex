\documentclass[a4paper, fleqn]{article}

\usepackage{clrscode}

\setlength\mathindent{0em}
\parskip = 7bp

\begin{document}

\section*{Solution to Exercise 5.1-2}

The implementation of \proc{Random}($a$, $b$) that only makes calls to
\proc{Random}(0,~1) is as follows:

\begin{codebox}
\Procname{$\proc{Random}(a, b)$}
\li  $n \gets \lceil\lg (b - a + 1)\rceil$
\li  \Repeat                                      \label{li:randab-repeat-begin}
         $r \gets 0$
\li      \Comment Generate an integer between 0 and $2^{\lceil\lg (b - a + 1)\rceil}$
                  inclusive, with \\ \>\>\>each such integer being equally likely
\li      \For $i \gets 0$ \To $n$                 \label{li:randab-for-begin}
\li          \Do
                $\mbox{the }i\mbox{th bit of }r \gets \proc{Random}(0, 1)$
             \End                                 \label{li:randab-for-end}
\li  \Until $r \le b - a$                         \label{li:randab-repeat-end}
\li  \Return $a + r$
\end{codebox}

The \kw{repeat} loop between line \ref{li:randab-repeat-begin} and
\ref{li:randab-repeat-end} ends if and only if the integer $r$ generated by the
\kw{for} loop between line \ref{li:randab-for-begin} and \ref{li:randab-for-end}
is between 0 and $b - a$ inclusive. Thus, each time the \kw{repeat} loop begins,
the probability for it to end is $p = (b - a)/2^{\lceil\lg (b - a + 1)\rceil}$,
and the expected number of iterations of the \kw{repeat} loop is
\[
t = \sum_{k = 1}^{+\infty}kp(1 - p)^{k - 1} = \frac{1}{p}.
\]
Note that the running time of the \kw{for} loop between line
\ref{li:randab-for-begin} and \ref{li:randab-for-end} is
$\Theta(n) = \Theta(\lceil\lg (b - a + 1)\rceil)$. Therefore, the expected running
time of \proc{Random}$(a,b)$ is $\Theta((\lceil\lg (b - a + 1)\rceil) / p)$.








\section*{Solution to Exercise 5.1-3}

\begin{codebox}
\Procname{$\proc{Random}$}
\li  \Repeat                                      \label{li:rand-repeat-begin}
         $r \gets 0$
\li      \Comment Generate an integer between 0 and 3 inclusive
\li      $\mbox{the 0th bit of } r \gets \proc{Biased-Random}$
                                                  \label{li:rand-gen-r-0}
\li      $\mbox{the 1st bit of } r \gets \proc{Biased-Random}$
                                                  \label{li:rand-gen-r-1}
\li  \Until $r = 1$ or $r = 2$                    \label{li:rand-repeat-end}
\li  \Return $r - 1$
\end{codebox}

In the procedure of \proc{Random} above, line \ref{li:rand-gen-r-0} and line
\ref{li:rand-gen-r-1} in the \kw{repeat} loop generate an integer $r$ between 0
and 3 inclusive, where 0, 1, 2, 3 being likely with the probabilities of
$(1 - p)^2, p(1 - p), p(1 - p)$ and $p^2$, respectively. That is, 1 and 2 are
equally likely. Similar as in the solution of Exercise 5.1-2, the \kw{repeat}
loop ends if and only if the generated $r$ is equal to 1 or 2. Thus, each time
the \kw{repeat} loop begins, the probability for it to end is $2p(1 - p)$. Let
$q = 2p(1 - p)$, then the expected number of iterations of  the \kw{repeat} loop is
\[
t = \sum_{k = 1}^{+\infty} kq(1 - q)^{k - 1} = \frac{1}{q} = \frac{1}{2p(1 - p)}.
\]
Therefore, the expected running time of \proc{Random} is
$\Theta(\frac{1}{2p(1 - p)})$.

\section*{Solution to Exercise 5.2-3}

Consider the expected value of one dice, we will define 6 variables associated
with the value the dice coming up. In particular, we have
\[
X_i = \mbox{I}\,\{\mbox{the dice comes up } i\} = \left\{ \begin{array}{ll}
      i & \mbox{if the dice comes up } i, \\
      0 & \mbox{if the dice comes up other values},
      \end{array} \right.
\]
where $i = 1, 2,\ldots, 6$.

Then the expected value of one dice is
\begin{eqnarray*}
\mbox{E}\,[X] & = & X_1 Pr\{X_1\} + X_2 Pr\{X_2\} + X_3 Pr\{X_3\} + \\
            &   & X_4 Pr\{X_4\} + X_5 Pr\{X_5\} + X_6 Pr\{X_6\} \\
  & = & 1 \cdot \frac{1}{6} + 2 \cdot \frac{1}{6} + 3 \cdot \frac{1}{6} +
        4 \cdot \frac{1}{6} + 5 \cdot \frac{1}{6} + 6 \cdot \frac{1}{6} \\
  & = & \frac{7}{2}
\end{eqnarray*}
Thus, the expected value of the sum of $n$ dice is $n \cdot \mbox{E}\,[X] = 7n/2$.





\section*{Solution to Exercise 5.2-4}

We will define $n$ variables related to whether or not each customer gets back
their own hat. In particular , we let $X_i$ be the indicator random variable
associated with the event in which the $i$th customer gets back his or her own
hat. Thus,
\begin{eqnarray*}
X_i & = & \mbox{I}\,\{\mbox{customer }i\mbox{ gets back his or her own hat}\} \\
    & = & \left\{\begin{array}{ll}
          1 & \mbox{if customer }i\mbox{ gets back his or her own hat}, \\
          0 & \mbox{if customer }i\mbox{ does not get back his or her own hat},
          \end{array}\right.
\end{eqnarray*}
and the number of customers that get back their own hat is
\[
X = X_1 + X_2 + \cdots + X_n.
\]
By Lemma 5.1, we have that
\begin{eqnarray*}
\mbox{E}\,[X_i] & = & \mbox{Pr}\,\{\mbox{customer } i
                      \mbox{ gets back his or her own hat}\} \\
                & = & 1/n
\end{eqnarray*}
Therefore, the expected number of customers that get back their own hat is
$\mbox{E}\,[X] = \sum_{i = 1}^n \mbox{E}\,[X_i] = 1$.






\section*{Solution to Exercise 5.2-5}

We will define $X_{ij}$ as the indicator random variable associated with the
event in which $A[i] > A[j]$ where $1 \le i < j \le n$, i.e., $(i,j)$ is an
inversion of $A$. Then there will be $n(n - 1)/2$ such $X_{ij}$ for an array
$A[1 \ldots n]$. Thus,
\[
X_{ij} = \mbox{I}\,\{(i,j) \mbox{ is an inversion of } A\} =
         \left\{\begin{array}{ll}
         1 & \mbox{if } (i,j) \mbox{ is an inversion of } A, \\
         0 & \mbox{if } (i,j) \mbox{ is not an inversion of } A,
         \end{array}\right.
\]
and the number of inversions in $A$ is
\[
X = \sum_{1 \le i < j \le n} X_{ij}.
\]
Consider the value of Pr\{$X_{ij} = 1$\}. Since the elements of $A$ form a
uniform random permutation of $\langle 1,2,\ldots,n \rangle$, for any fixed
value of $A[i] = k$, the case of $A[i] < A[j]$ occurs with probability $n - k$.
Therefore, by Lemma 5.1 we have that
\[
\mbox{E}\,[X_{ij}] = \mbox{Pr}\,\{X_{ij} = 1\} =
\frac{\sum_{k = 1}^n (n - k)}{n(n - 1)} = \frac{1}{2},
\]
and by linearity of expectation we have that
\[
\mbox{E}\,[X] = \sum_{1 \le i < j \le n} \mbox{E}\,[X_{ij}] = \frac{n(n - 1)}{4}.
\]
Thus the expected number of inversions is $n(n - 1)/4$.






\section*{Solution to Exercise 5.3-1}

\begin{codebox}
\Procname{$\proc{Random-In-Place-2}(A)$}
\li  $n \gets \id{length}[A]$
\li  swap $A[1] \leftrightarrow A[\proc{Random}(1,n)]$
\li  \For $i \gets 2$ \To $n$    \Comment Note that $i$ starts from 2
                                                \label{li:r-per-for-begin}
\li      \Do
             swap $A[i] \leftrightarrow A[\proc{Random}(i,n)]$
                                                \label{li:r-per-for-end}
     \End
\end{codebox}

The associated loop invariant is nearly the same as that in \proc{Random-In-Place},
except that the \kw{for} loop is now in line
\ref{li:r-per-for-begin}--\ref{li:r-per-for-end}. We only need to modify the
intialization of the proof of Lemma 5.5. Currently, before the first loop iteration,
we have $i = 2$. The loop invariant says that for each possible 1-permutation, the
subarray $A[1 \twodots 1]$ contains this 1-permutation with probability
$(n - i + 1)! / n! = (n - 1)! / n! = 1/n$. This holds because $A[1]$ is chosen
randomly from the $n$ values in positions $A[1 \twodots n]$.






\section*{Solution to Exercise 5.3-2}

No, some other permutations can not be produced besides the identity permutation.
For any $A[i]$ before the code runs, it will not stay in position $i$ any more
when the code completes. Therefore, the permutations where some $A[i]$ still stays
in position $i$ can not be produced. For example, permutation $\{1, 3, 2\}$ can
not be produced by this code for $A = \{1, 2, 3\}$.






\section*{Solution to Exercise 5.3-4}

For each $A[i]$ and any position $j$, if $j > i$, then $A[i]$ will wind up in
position $j$ in $B$ if and only if $\id{offset} = j - i$; otherwise $A[i]$ will
wind up in position $j$ in $B$ if and only if $\id{offset} = n + j - i$. Since
\id{offset} is the returned value of \proc{Random}$(1,n)$ in this algorithm,
$A[i]$ will wind up in any position $j$ in $B$ with the probability of $1/n$.
However, some permutations of $A$ can't be produced by \proc{Permute-By-Cyclic},
such as $B = \{A[1], A[2], A[4], A[3], \ldots\}$. Thus, the resulting permutation
is not uniformly random.







\section*{Solution to Exercise 5.3-5}

Let $E_i$ be the event that $P[i]$ is unique, then we have
\begin{eqnarray*}
& & \mbox{Pr}\,\{\mbox{all elements of } P \mbox{ are unique}\}  \\
& = & \mbox{Pr}\,\{E_1 \cap E_2 \cap \ldots \cap E_n \} \\
& = & \mbox{Pr}\,\{E_1\} \cdot \mbox{Pr}\,\{E_2|E_1\} \cdots \mbox{Pr}\,\{E_n | E_1 \cap E_2 \cap \ldots \cap E_{n-1}\} \\
& = & \frac{n^3}{n^3} \cdot \frac{n^3 - 1}{n^3} \cdot \frac{n^3 - 2}{n^3} \cdots \frac{n^3 - (n - 1)}{n^3} \\
& \geq & \left(\frac{n^3 - n}{n^3}\right)^{n-1} \\
& = & \left(1 - \frac{1}{n^2}\right)^{n-1} \\
& \geq & 1 - \frac{n-1}{n^2} \hspace{1cm}  \mbox{since } (1 - a)(1 - b) \geq (1 - (a + b)) \mbox{ if } a, b \geq 0 \\
& \geq & 1 - \frac{n}{n^2} \\
& = & 1 - \frac{1}{n}.
\end{eqnarray*}









\section*{Solution to Exercise 5.3-6}

I don't know. Maybe the only way is to prevent the calling to \proc{Random} from
generating two or more identical priorities.








\section*{Solution to Exercise 5.4-1}

Suppose there are $k$ people in the room, then the probability that nobody's
birthday is the same as mine is $(n - 1)^k/n^k$ where $n = 365$. So the probability
that someone has the same birthday as I do is $p_1 = 1 - (n - 1)^k/n^k = 1 - (364/365)^k$.
Let $p_1 \geq 1/2$ then we have $k \geq \ln(1/2)/\ln(364/365)$, i.e., $k \geq 253$.

The probability that at most one person has a birthday on a given day is
\begin{eqnarray*}
p'_2 & = & \left(\frac{n - 1}{n}\right)^k +
\left(\!\!\begin{array}{cc}k \\ 1\end{array}\!\!\right)
\cdot \frac{1}{n} \cdot \left(\frac{n - 1}{n}\right)^{k - 1} \\
& = & \left(\frac{n - 1}{n}\right)^{k - 1} \cdot \frac{n - 1 + k}{n}.
\end{eqnarray*}
Note that this is a monotonically decreasing function of $k$ when $k$ is
sufficiently large. Using a trial-and-error approach, we know that $p'_2 = 0.51$
when $k = 606$ and $p'_2 = 0.50$ when $k = 607$. Therefore, we may informally
infer that $p'_2 \leq 1/2$ when $k \geq 607$. Thus, there should be at least 607
people so that the probability that at least two people have a birthday on July 4
is greater than $1/2$.








\section*{Solution to Exercise 5.4-2}

Actually, this problem is equivalent to the birthday paradox, with the number of
people replaced by the number of balls, and the number of days in a year replaced
by the number of bins so that $n = b$. For each pair $(i,j)$ of the $k$ balls, we
define the indicator random variable $X_{ij}$, for $1 \leq i < j \leq k$, by
\begin{eqnarray*}
X_{ij} & = & \mbox{I}\,\{\mbox{ball } i \mbox{ and ball } j \mbox{ are tossed in the same bin}\} \\
       & = & \left\{ \begin{array}{ll}
       1 & \mbox{ball } i \mbox{ and ball } j \mbox{ are tossed in the same bin}, \\
       0 & \mbox{otherwise}.
       \end{array}\right.
\end{eqnarray*}
By equation (5.7), the probability that two balls are tossed in the same bin is
$1/b$, and thus by Lemma 5.1, we have
\begin{eqnarray*}
\mbox{E}\,[X_{ij}] & = & \mbox{Pr}\,\{\mbox{ball } i \mbox{ and ball } j \mbox{ are tossed in the same bin}\} \\
                   & = & 1/b.
\end{eqnarray*}
Letting $X$ be the random variable that counts the number of pairs of balls being
tossed in the same bin, we have
\[
X = \sum_{i = 1}^k \sum_{j = i + 1}^k X_{ij}.
\]
Taking expectations of both sides and applying linearity of expectation, we obtain
\begin{eqnarray*}
\mbox{E}\,[X] & = & \mbox{E}\,\left[\sum_{i = 1}^k \sum_{j = i + 1}^k X_{ij}\right] \\
              & = & \sum_{i = 1}^k \sum_{j = i + 1}^k \mbox{E}\,[X_{ij}] \\
              & = & \left(\!\begin{array}{c} k \\ 2 \end{array}\!\right)  \frac{1}{b}\\
              & = & \frac{k(k - 1)}{2b}.
\end{eqnarray*}
When $k(k - 1) \geq 2b$, therefore, the expected number of pairs of balls being
tossed in the same bin is at least 1. Thus, if we have at least $\sqrt{2b} + 1$
balls being tossed, we can expect at least one of the bins contains two balls.







\section*{Solution to Exercise 5.4-3}

Mutual independence is very important for the analysis of birthday paradox,
while pairwise independence is not sufficient. Let's consider a special case,
in which we know that any $b_{i+2} = \max(b_{i}, b_{i+1})$, where $i \equiv 1$
(mod 3). In this case, the birthdays are pairwise independent. However, any
triple of $\langle b_{i}, b_{i+1}, b_{i+2} \rangle$ where $i \equiv 1$ (mod 3)
are not mutually independent. Now, just 3 people is enough to ensure that two
of them were born on the same day.







\section*{Solution to Exercise 5.4-4}

We can use indicator random variables to solve this problem simply as in
subsection 5.4.1. For each triple $(s,t,u)$ of the $k$ people, we define the
indicator random variable $X_{stu}$, for $1 \leq s < t < u \leq k$, such that
$X_{stu} = 1$ when people $s$, $t$ and $u$ have the same birthday, and $X_{stu} 
= 0$ when they don't. Then we have $\mbox{E}[X_{stu}] = \mbox{Pr}\{X_{stu = 
1}\} = 1/n^2$. Also let $X$ be the random variable that counts the number of 
tiples of individuals having the same birthday, we obtain
\begin{eqnarray*}
\mbox{E}[X] & = & \sum_{1 \leq s < t < u \leq k} \mbox{E}[X_{stu}] \\
            & = & \left(\!\begin{array}{c}k \\
                  3\end{array}\!\right) \frac{1}{n^2} \\
            & = & \frac{k(k - 1)(k - 2)}{6n^2}
\end{eqnarray*}
Since $k(k-1)(k-2) \geq (k-2)^3$, letting $(k-2)^3 \geq 6n^2$ gaurantees that 
$k(k-1)(k-2) \geq 6n^2$ so that the expected number of triples of people with 
the same birthday is at least 1. Thus, if we have at least $\sqrt[3]{6n^2} + 2$ 
individuals in a room, we can expect at least three to have the same birthday. 
For $n = 365$, if $k = 95$, the expected number of triples  with the same birthday 
is $(95 \cdot 94 \cdot 93)/(6 \cdot 365^2) \approx 1.0390$. Thus, with at least 
95 people, we expect to find at least one patching triple of birthdays.







\section*{Solution to Exercise 5.4-5}

The probability is $\frac{n!}{n^k (n-k)!}$. In the birthday paradox, this is the 
probability that $k$ people have distinct birthdays.







\section*{Solution to Exercise 5.4-7}

Similar as in subsection 5.4.3, we partition the $n$ coin flips into at least 
$\lfloor n / \lfloor \lg n - 2\lg\lg n \rfloor \rfloor$ groups of $\lfloor 
\lg n - 2\lg\lg n \rfloor$ consecutive flips, and we bound the probability that 
no group comes up all heads. By equation (5.9), the probability that the group 
starting in position $i$ comes up all heads is
\begin{eqnarray*}
\mbox{Pr}\{A_{i,\lfloor \lg n - 2\lg\lg n \rfloor}\} & = & 1 / 2^{\lfloor \lg n - 2\lg\lg n \rfloor} \\
& \leq & (\lg^2 n) / n.
\end{eqnarray*}
The probability that a streak of heads of length at least $\lfloor \lg n - 
2\lg\lg n \rfloor$ does not begin a position $i$ is therefore at most $1 - 
(\lg^2 n) / n$. Since the $\lfloor n / \lfloor \lg n - 2\lg\lg n \rfloor 
\rfloor$ groups are formed from mutually exclusive, independent coin flips, 
the probability that every one of these groups fails to be a streak of length 
$\lfloor \lg n - 2\lg\lg n \rfloor$ is at most
\begin{eqnarray*}
(1 - (\lg^2 n) / n)^{\lfloor n / \lfloor \lg n - 2\lg\lg n \rfloor \rfloor} & \leq &
(1 - (\lg^2 n) / n)^{n / \lfloor \lg n - 2\lg\lg n \rfloor - 1} \\
& \leq & (1 - (\lg^2 n) / n)^{n / (\lg n - 2\lg\lg n) - 1} \\
& \leq & (1 - (\lg^2 n) / n)^{n / \lg n} \\
& \leq & e^{-(\lg^2 n) / n \cdot n / \lg n} \\
&   =  & e^{\lg n} \\
& \leq & 2^{\lg n} \\
&   =  & 1 / n.
\end{eqnarray*}
For this argument, we used the fact that $n / (\lg n - 2\lg\lg n ) - 1 \geq n 
/ \lg n$ for sufficiently large $n$ in step 3, and inequality (3.11) in step 4.

According to section 2 in my notes of this chapter, the probability that no 
streak longer than $\lfloor \lg n - 2\lg\lg n \rfloor$ consecutive heads occurs 
is less than the probability that every $\lfloor \lg n - 2\lg\lg n \rfloor$ 
group fails to be a streak of consecutive heads. Thus, the probability that no 
streak longer than $\lfloor \lg n - 2\lg\lg n \rfloor$ consecutive heads is less 
than $1/n$.

For more rigorousness, we will prove the fact that $n / (\lg n - 2\lg\lg n ) - 
1 \geq n / \lg n$ for all $n > 16$ which we used in step 3. First of all, we have
\begin{eqnarray*}
\frac{n}{\lg n - 2\lg\lg n} - 1 - \frac{n}{\lg n}
& = & \frac{(n - \lg n + 2\lg\lg n)\lg n - n (\lg n - 2\lg\lg n)}{(\lg n - 2\lg\lg n)\lg n} \\
& = & \frac{2n\lg\lg n - \lg^2 n + 2\lg n\lg\lg n}{(\lg n - 2\lg\lg n)\lg n}.
\end{eqnarray*}
Assuming that the factors in denominator are both positive, that is, $n > 16$, 
we shall only consider the numerator, which is $2n\lg\lg n - \lg^2 n + 2\lg n
\lg\lg n$. It is easy to judge that $2n\lg\lg n$ grows faster than $\lg^2 n$, because polynomials grow faster than logarithms, and $2n\lg\lg n$ grows faster 
than polynomial $n$. Since $2n\lg\lg n = 64 > \lg^2 n = 16$ when $n = 16$, we 
have $2n\lg\lg n \geq \lg^2 n$ for all $n > 16$, then the numerator $2n\lg\lg n 
- \lg^2 n + 2\lg n\lg\lg n \geq 0$, and $n / (\lg n - 2\lg\lg n ) - 1 - n / 
\lg n \geq 0$, i.e., $n / (\lg n - 2\lg\lg n ) - 1 \geq n / \lg n$ when $n > 16$.







\section*{Solution to Problem 5-2}

\begin{enumerate}
\renewcommand{\labelenumi}{\itshape \bfseries \alph{enumi}.}

\item % a
Here's the pseudocode that implement the randomized strategy:
\begin{codebox}
\Procname{$\proc{Random-Search}(A,x)$}
\li $n \gets \id{length}[A]$
\li create array $B[1 \twodots n]$
\li \For $i \gets 1$ \To $n$
\li     \Do
            $B[i] \gets 0$
        \End
\li $\id{count} \gets 0$ \>\>\>\> \Comment Number of indices into $A$ that have been picked.
\li \Repeat
        $i \gets \proc{Random}(1,n)$
\li     \If $A[i] = x$
\li         \Then
                \Return $i$
            \End
\li     \If $B[i] = 0$
\li         \Then
                $B[i] \gets 1$
\li             $\id{count} \gets \id{count} + 1$
            \End
\li \Until $\id{count} = n$
\li \Return \const{nil}
\end{codebox}

\item % b
This problem is equivalent to that in the problem of balls and bins, which is 
\textit{how many balls must one toss, on the average, until a given bin contains 
a ball}. Here, the given bin is substituted by the index to $A[i] = x$, while 
the tosses of balls are substituted by the action of picking indices. Thus, the 
expected number of indices into $A$ that must be picked before $x$ is found and 
\proc{Random-Search} terminates is $1/(1/n) = n$.

\item % c
The expected number of indices that must be picked is $1/(k/n) = n/k$.

\item % d
This problem is equivalent to that in the problem of balls and bins, which is 
\textit{how many balls must one toss until every bin contains at least one ball}. 
Thus the expected number of indices that must be picked is $n (\ln n + O(1))$.

\item % e
Since the probability of $x$ to be in any position $i$ where $1 \leq i \leq n$ 
is $1/n$, the expected running time of \proc{Deterministic-Search} is $1 \cdot 
1/n + 2 \cdot 1/n + \cdots + n \cdot 1/n = (n + 1)/2$. The worst-case running 
time is $n$.

\item % f
For the k indices that point to the elements of $x$, the probability of the 
leftmost index being $j$ is 
\[
p_j = \frac{\left(\!\begin{array}{c}n-j \\ k-1\end{array}\!\right)}{\left(\!\begin{array}{c}n \\ k\end{array}\!\right)}
\]
where $1 \leq j \leq n - k + 1$. The expected running time is $\sum_{j=1}^{n-k+1} j p_j$.

The worst-case running time is $n - k + 1$.

\item % g
The expected running time and the worst-case running time are both $n$.

\item % h
After permuting the input array, we have achieved a situation identical to that 
of \proc{Deterministic-Search}. Thus, the running times are all the same to those 
of \proc{Deterministic-Search}.

\item % i
\proc{Deterministic-Search}, of course. \proc{Random-Search} may results in 
never terminating situation, while \proc{Scramble-Search} takes time permuting 
the input array.

\end{enumerate}

\end{document}
