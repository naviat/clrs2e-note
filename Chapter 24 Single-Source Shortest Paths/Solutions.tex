\documentclass[a4paper, fleqn]{article}
\usepackage{amsmath}
\usepackage{array}
\usepackage{clrscode}
\usepackage{enumitem}
\usepackage{hyperref}
\usepackage[margin=1in]{geometry}
\usepackage{color}
\setlength\mathindent{0em}
\parskip = 5bp

\title{Solutions for Chapter 24}
\author{Zhixiang Zhu  \\\href{mailto:zzxiang21cn@hotmail.com}{zzxiang21cn@hotmail.com}}

\begin{document}

\maketitle

\section*{Solution to Exercise 24.1-2}

Suppose there's an {\bf acyclic} path from $s$ to $v$ being $\langle v_0, v_1, \ldots, v_k
\rangle$, where $v_0 = s$ and $v_k = v$. Similarly with the proof of lemma 24.2, the path
has at most $|V| - 1$ edges, and so $k \leq |V| - 1$. Each of the $|V| - 1$ iterations of
the \For loop of lines 2-4 relaxes all $E$ edges. Among the edges relaxed in the $i$th
iteration, for $i = 1, 2, \ldots, k$, is $(v_{i - 1}, v_i)$, and $d[v_i]$ becomes finite.
Therefore, $d[v_k] = d[v]$ will become finite no later than the $k$th iteration.

The opposite direction can be proved by appealing to the no-path property (Corollary
24.12).

\section*{Solution to Exercise 24.1-3}

For any pair of vertices $u$, $v \in V$, the shortest path between $u$ and $v$ having the
minimum number of edges must be acyclic. So we can get the idea from the proof of Lemma
24.2. Consider any vertex $v$ that is reachable from $s$, and let $p = \langle v_0, v_1,
\ldots, v_k \rangle$, where $v_0 = s$ and $v_k = v$, be any acyclic shortest path from $s$
to $v$. Path $p$ has at most $\min(m, |V|-1)$ edges, i.e. $k \leq \min(m, |V|-1)$. Among
the edges relaxed in the $i$th pass, for $i = 1, 2, \ldots, k$, is $(v_{i-1}, v_i)$. By
the path-relaxation property, no relaxation will occur on $p$ after the $k$th pass (with
the precondition that $G$ has no negative-weight cycles. So we can make a simple change to
the \For loop of lines 2--4 of \proc{Bellman-Ford}: if no edge is relaxed during the pass,
the loop can terminate. Also, lines 5--8 of \proc{Bellman-Ford} can be removed because
this simple change causes the algorithm to lose the ability to detect negative-weight
cycles.

\end{document}
