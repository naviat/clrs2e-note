\documentclass[a4paper, fleqn]{article}
\usepackage{clrscode}
\setlength\mathindent{0em}
\parskip = 7bp
\begin{document}

\section*{Solution to Exercise 14.1-8}

For the clearness and convenience of description, the following
assumptions are given:
\begin{enumerate}
\item
For a given circle, a radius is chosen as the axis.

\item
A point on the circle is defined by the angle of clockwise rotation
from the axis to the radius connecting the point and the center. The
endpoint of the axis on the circle is $0^\circ$. Thus, a point is
defined as a real number between 0, inclusive, and 360, exclusive.

\item
A chord $c$ on the circle is defined by two endpoint: $\id{start}[c]$,
which is the smaller one, and $\id{end}[c]$, which is the larger one.
\end{enumerate}

The key data structures are two order-statistic trees $S$ and $L$,
where the key of each node is a real number, which is an endpoint of
chord. $S$ contains the start endpoint of each chord, while $L$
contains the end endpoint of each chord. However, each node $x$ in $S$
contains one more field, \id{other-end}[$x$], which is a pointer to
the node $y$ in $L$, where \id{key}[$x$] and \id{key}[$y$] is the two
endpoints of a chord among the $n$ input chords.

Also, a procedure \proc{Get-Smaller-Num}$(T,k)$ is used, which takes a
binary search tree $T$ and a key $k$ as its input, and returns the
number of nodes in $T$ whose keys are smaller than $k$. Since this
procedure can be performed as a slightly modified search operation,
and takes a red-black tree as its input in the algorithm followed, its
running time is $O(\lg n)$.

The procedure \proc{Get-Intersect-Num} takes an array of chords as its
input and returns the number of pairs of chords that intersect inside
the circle:

\begin{codebox}
\Procname{\proc{Get-Intersect-Num}$(A)$}
\li $n \gets \id{length}[A]$
\li Create two empty order statistic trees $S$ and $L$.
\li \For $i \gets 1$ \To $n$
\li   \Do
        Create a node $x$ for $S$ and another node $y$ for $L$.
\li     $\id{key}[x] \gets \id{start}[A[i]]$
\li     $\id{size}[x] \gets 1$
\li     $\id{other-end}[x] \gets y$
\li     \proc{RB-Insert}$(S,x)$
\li     $\id{key}[y] \gets \id{end}[A[i]]$
\li     $\id{size}[y] \gets 1$
\li     \proc{RB-Insert}$(L,y)$
      \End
\li $\id{count} \gets 0$
\li \For each node $x$ of $S$ during the in-order walk of $S$
\li   \Do
        $y \gets \id{other-end}[x]$
\li     $\id{count} \gets \id{count} + 
          \proc{Get-Smaller-Num}(S, \id{key}[y]) - $
\zi         \proc{OS-Rank}$(L,y)$
                          \label{li:inter-num-get-count}
\li     \proc{RB-Delete}$(S,x)$
\li     \proc{RB-Delete}$(L,y)$
      \End
\li \Return \id{count}
\end{codebox}

Note that the procedures of \proc{RB-Insert} and \proc{RB-Delete} used
here are the modified versions in Section 14.1 maintaining the size of
subtree rooted at each node.

Let's argue the correctness of the algorithm first. For any chord $i$,
We define $a_i$ as its start endpoint, while $b_i$ as its end
endpoint. Any chords $i$ and $j$ intersect if and only if $a_i < a_j <
b_i < b_j$ or $a_j < a_i < b_j < b_i$. In the first case, chord $j$
intersects with $i$ in \emph{forward} direction, since chord $j$ has
its start endpoint between $a_i$ and $b_i$, but not its end endpoint.

The basic idea of the algorithm is that for each chord $i$, only the
number of chords which intersect with $i$ in forward direction is
counted. The first \kw{for} loop sorts the $n$ chords according to
their start endpoints. Then the second \kw{for} loop traverses the $n$
chords in this sorted order and count the number of chords that
intersect with the current chord in forward direction in each
iteration. After a chord is processed, its endpoints are deleted from
the two order-statistic red-black trees. This guarantees that the call
of \proc{Get-Smaller-Num} in line \ref{li:inter-num-get-count} returns
one plus the number of start endpoints lying between the current
chord, while the call of \proc{OS-Rank} in the same line returns one
plus the number of chords whose start endpoints and end endpoints all
lie between the current chord (so these chords do not intersect with
the current chord).

The two \kw{for} loops both iterate for $n$ times, with each iteration
takes $O(\lg n)$-time. Therefore, the whole algorithm takes running
time of $O(n \lg n)$.

Note: The implementation of algorithm indicated in the file of
``Problem Set \#4 Solutions.pdf'' may not work correctly. In that
implementation, all the endpoints are traversed in sorted order
regardless of being a start endpoint or an end endpoint. A start
endpoint is deleted only when its corresponding end endpoint is
met. Consider three chords with the order of endpoints being $a_1 <
a_2 < a_3 < b_2 < b_3 < b_1$. When $b_2$ is met, $a_2$ is deleted,
then it is supposed that $a_3$ is the only endpoint in the tree
indicating that only chord 3 intersect with chord 2 in the forward
direction. However, by the implementation, $a_1$ is still in the tree
because $b_1$ has not been met! Therefore, a wrong number of 2 is
actually counted when $b_2$ is met.






\section*{Solution to Exercise 14.3-3}

As we travel down the tree, if the current node $x$ does not overlap
the query interval $i$, then we should go down to either the left or
right child, which is unambiguous. However, if $x$ overlaps $i$, the
node $x$ can not be returned immediately because we are not sure
whether it is the one having the minimum low endpoint. Therefore, we
still need to check the left subtree of $x$ to see that whether
there's another node that overlaps $i$. The right subtree doesn't need
to be check because the low endpoints in it are all smaller than that
of $x$, which is trivial.

Based on this idea, the iterative version of
\proc{Min-Interval-Search} is given as below:

\begin{codebox}
\Procname{\proc{Min-Interval-Search}$(T,i)$}
\li $x \gets \id{root}[T]$
\li $\id{min} \gets \id{nil}[T]$
\li \While $x \neq \id{nil}[T]$
\li   \Do
        \If $i$ overlaps \id{int}[$x$]
\li       \Then
            $\id{min} \gets x$
\li         $x \gets \id{left}[x]$
\li     \ElseIf $\id{left}[x] \neq \id{nil}[T]$
            and $\id{max}[\id{left}[x]] \geq \id{low}[i]$
\li       \Then
            $x \gets \id{left}[x]$
\li     \ElseNoIf
            $x \gets \id{right}[x]$
        \End
      \End
\li \Return \id{min}
\end{codebox}
The call \proc{Min-Interval-Search}$(T,i)$ takes $O(\lg n)$ time,
since each iteration of the \kw{while} loop goes one node lower the
tree, and the height of the tree is $O(\lg n)$.






\section*{Solution to Exercise 14.3-4}

A simple idea to list all intervals in $T$ that overlap $i$ is calling
\proc{Interval-Search$(T,i)$} again and again with each time the node
returned overlapping $i$ deleted from $T$, until \id{nil}[$T$] is
returned. If there are $k$ qualified intervals in the tree, then this
method takes $O(k \lg n)$ running time.

If each node is traversed no more than constant times, then the
running time of the listing procedure will also be $O(n)$, so that all
intervals in $T$ that overlap $i$ can be listed in $O(\min(n, k \lg
n))$ time.

The procedure \proc{Interval-Search-All} given below, which takes an
interval tree $T$ and an interval $i$ as its input and returns a list
of nodes which overloap $i$, accomplishes this goal.

\begin{codebox}
\Procname{\proc{Interval-Search-All}$(T,i)$}
\li create an empty list $L$
\li create an empty stack $S$
\li \If $T$ is not empty
\li   \Then \proc{Push}$(S, \id{root}[T])$ \End
\li \Repeat                        \label{li:ISA-outer-repeat-begin}
\li     \While not \proc{Stack-Empty}$(S)$
           and $\id{max}[S[\id{top}[S]]] < \id{low}[i]$
                                           \label{li:ISA-while-begin}
\li       \Do \proc{Pop}$(S)$ \End         \label{li:ISA-while-end}
\li     \If not \proc{Stack-Empty}$(S)$
\li       \Then
            \Repeat                 \label{li:ISA-inner-repeat-begin}
\li           $x \gets y \gets \proc{Pop}(S)$
\li           \If $\id{right}[x] \neq \id{nil}[T]$
\li             \Then \proc{Push}$(S, \id{right}[x])$ \End
\li           \If $\id{left}[x] \neq \id{nil}[T]$
                and $\id{max}[\id{left}[x]] \geq \id{low}[i]$
\li             \Then
                  \proc{Push}$(S, \id{left}[x])$
\li               $x \gets \id{left}[x]$
\li             \Else
                  $x \gets \id{right}[x]$
                \End
\li         \Until $x = \id{nil}[T]$ or $i$ overlaps \id{int}[$y$]
                                      \label{li:ISA-inner-repeat-end}
\li         \If $i$ overlaps \id{int}[$y$]
\li           \Then add $y$ to $L$ \End
          \End
\li \Until \proc{Stack-Empty}$(S)$ or $i$ does not overlap \id{int}[$y$]
                                      \label{li:ISA-outer-repeat-end}
\li \Return $L$
\end{codebox}

In the procedure of \proc{Interval-Search-All}, each iteration of the
outer \kw{repeat} loop of lines
\ref{li:ISA-outer-repeat-begin}--\ref{li:ISA-outer-repeat-end}
performs a search in $T$ to find a new (not found in the previous
iterations) interval that overlaps $i$. If no such interval is found,
then the outer \kw{repeat} loop ends since all intervals in $T$ that
overlap $i$ have been found, so that the whole procedure also ends.

The searching procedure in each iteration of the outer \kw{repeat}
loop is the same as that of \proc{Interval-Search}, except that it
does not always begin from the root of $T$. This is based on that the
nodes having been traversed in the previous iterations can be skipped.

The skip is accomplished by an auxiliary stack $S$, which initially
contains the root of $T$ if $T$ is not empty. As we perform downward
search in $T$ in each iteration of the inner \kw{repeat} loop of lines
\ref{li:ISA-inner-repeat-begin}--\ref{li:ISA-inner-repeat-end}, we pop
the head of the stack, which is also a node of interval tree, to check
its interval, then push its children to the stack if they
exist. Therefore, the inner \kw{repeat} loop ends with the roots of
untraversed subtrees stored in the stack. Notice that these subtrees
are disjoint. Also, among the roots of these subtrees, those which are
closer to the head in the stack appear left to other roots in the
tree, which are further from the head in the stack. Therefore, at the
start of the next iteration of the outer \kw{repeat} loop of lines
\ref{li:ISA-outer-repeat-begin}--\ref{li:ISA-outer-repeat-end}, we can
skip the node traversed previously, and select one of the subtrees not
traversed to perform our downward search again, in which if there is
no interval overlapping $i$, then there is no interval overlapping $i$
in other subtrees. Since each node of $T$ is traversed at most twice,
of which the first is being pushed into $S$, the second is being pop
from $S$ to check its interval, the running time of
\proc{Interval-Search-All}$(T,i)$ is $O(n)$, where $n$ is the number
of nodes in $T$.

The selection of roots of untraversed subtrees is performed by the
\kw{while} loop of lines
\ref{li:ISA-while-begin}--\ref{li:ISA-while-end}, of which the
priciple is the same as the proof of correctness of line 4 of
\proc{Interval-Search}$(T,i)$. Because we replace just the head of $S$
with its children in each iteration of the inner \kw{repeat} loop of
lines \ref{li:ISA-inner-repeat-begin}--\ref{li:ISA-inner-repeat-end},
each node in $S$ must be at different depth in $T$, except the head of
$S$ and its successor. Therefore, the length of $S$ is $O(\lg n)$, so
that in each iteration of the outer \kw{repeat} loop of lines
\ref{li:ISA-outer-repeat-begin}--\ref{li:ISA-outer-repeat-end}, the
running time of the \kw{while} loop of lines
\ref{li:ISA-while-begin}--\ref{li:ISA-while-end} is also $O(\lg
n)$. The inner \kw{repeat} loop of lines
\ref{li:ISA-inner-repeat-begin}--\ref{li:ISA-inner-repeat-end} acts
exactly the same as \proc{Interval-Search}, except that it doesn't
always begin from the root of $T$. Therefore, the running time of the
inner \kw{repeat} loop is $O(\lg n)$. The outer \kw{repeat} loop
iterate for $k+1$ times where $k$ is the number of intervals that
overlap $i$ in $T$. Therefore, the running time of the whole procedure
\proc{Interval-Search-All}$(T,i)$ is also $O(k \lg n)$. Thus, the
running time of \proc{Interval-Search-All} is $O(\min(n, k \lg
n))$. What's more, this algorithm also leaves the input tree $T$
unmodified.

The \proc{Interval-Search-All} procedure can be implemented in a more
compact version expanding the inner \kw{repeat} loop, which is given
below:

\begin{codebox}
\Procname{\proc{Interval-Search-All}$'(T,i)$}
\li create an empty list $L$
\li create an empty stack $S$
\li \If $T$ is not empty
\li   \Then \proc{Push}$(S, \id{root}[T])$ \End
\li \Repeat
\li     \While not \proc{Stack-Empty}$(S)$
           and $\id{max}[S[\id{top}[S]]] < \id{low}[i]$
                                         \label{li:ISA'-while-begin}
\li       \Do \proc{Pop}$(S)$ \End       \label{li:ISA'-while-end}
\li     \If not \proc{Stack-Empty}$(S)$
\li       \Then
            $x \gets y \gets \proc{Pop}(S)$
\li         \If $\id{int}[y]$ overlaps $i$
\li           \Then insert $y$ into $L$ \End
\li         \If $\id{right}[x] \neq \id{nil}[T]$
\li           \Then \proc{Push}$(S, \id{right}[x])$ \End
\li         \If     $\id{left}[x] \neq \id{nil}[T]$
                and $\id{max}[\id{left}[x]] \geq \id{low}[i]$
\li           \Then
                \proc{Push}$(S, \id{left}[x])$
\li             $x \gets \id{left}[x]$
\li           \Else
                $x \gets \id{right}[x]$
              \End
          \End
\li \Until \proc{Stack-Empty}$(S)$ or
           ($x$ = \id{nil}[$T$] and $y$ doesn't overlap $i$)
\end{codebox}

This version will be more efficient in some situations, since it check
whether the subtrees rooted at the children of the current node should
be discarded immediately whenever we replace the head of $S$ with its
tree children. This check is performed by the \kw{while} loop of lines
\ref{li:ISA'-while-begin}--\ref{li:ISA'-while-end}. Actually, this
\kw{while} loop iterates at most twice in each iteration of the
\kw{repeat} loop, because at the start of the \kw{while} loop, the
only possible nodes in $S$ whose \id{max} attributes are smaller than
\id{low}[$i$] are the heading two nodes if the node traversed in the
previous iteration of the \kw{repeat} loop has two children, or just
the head of $S$ if it has only one child, or none if it has no
children.






\section*{Solution to Exercise 14.3-5}

Only the insert procedure needs to be modified so that when we perform
an in-order walk on an interval tree, the output intervals is in sorted
order according to their low endpoints as the first sorting keys and
their hight endpoints as the second sorting keys.






\section*{Solution to Exercise 14.3-6}

We can use two augmented red-black tree $T1$ and $T2$ to maintain this
dynamic set $Q$. The keys in $T1$ are the numbers in $Q$, while the
keys in $T2$ are the gaps between neighboring numbers in $Q$. As in
the example given in the exercise, the keys in $T2$ are 4 (= 5 - 1), 4
(= 9 - 5), 6 (= 15 - 9), 3 (= 18 - 15) and 4 (= 22 - 18). Each node
$x$ in $T1$ is augmented by an extra attribute:
\id{successor-gap}$[x]$, which points to a node in $T2$, whose key is
the gap between \id{key}$[x]$ and the key of the successor of $x$.  In
the example given in the exercise, if \id{key}$[x]$ is 9, then
$\id{key}[\id{successor-gap}[x]]$ is 6. The \id{successor-gap} field
of the maximal node in $T1$ points to \const{nil}. Meanwhile, $T2$ is
augmented by an extra attribute $\id{min}[T2]$, which is the minimal
key in $T2$.

Let $n$ be the size of $Q$ at anytime. Then the number of nodes in
$T1$ is also $n$, while the number of nodes in $T2$ is $n-1$.

The \proc{Insert} operation first inserts a new node into $T1$, which
runs in $O(\lg n)$ time. After the new node is inserted, the the
\id{successor-gap} field of its predecessor (if it exists) will have
to be updated by deleting one relevant node and inserting two new
nodes in $T2$.  This also runs in $O(\lg n)$ time. If one of the keys
of the two new node inserted in $T2$ is smaller than $\id{min}[T2]$,
then $\id{min}[T2]$ should also be updated, which takes $O(1)$ time.
The whole \proc{Insert} operation runs in $O(\lg n)$ time.

The \proc{Delete} operation first updates the \id{successor-gap} field
of the predecessor of the node $x$ to be deleted from $T1$. This is
performed by deleting two relevant nodes (the one pointed to by
$\id{successor-gap}[x]$ and the one pointed to by the
\id{successor-gap} field of $x$'s predecessor) and inserting a new node
in $T2$, which runs in $O(\lg n)$ time. Then, the node in $T1$ is also
deleted, which again runs in $O(\lg n)$ time. If the minimal node in
$T2$ is deleted, then $\id{min}[T2]$ will also have to be updated,
which runs $O(\lg n)$ time. Therefore, the \proc{Delete} operation
also runs in $O(\lg n)$ time.

The \proc{Search} operation performs ordinary \proc{Tree-Search} on
$T1$, which takes $O(\lg n)$.

The \proc{Min-Gap} operation simply returns the value of
$\id{min}[T2]$, which runs in $O(1)$ time.

\end{document}
